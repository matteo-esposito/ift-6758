\documentclass[11pt]{article}

    \usepackage[breakable]{tcolorbox}
    \usepackage{parskip} % Stop auto-indenting (to mimic markdown behaviour)
    
    \usepackage{iftex}
    \ifPDFTeX
    	\usepackage[T1]{fontenc}
    	\usepackage{mathpazo}
    \else
    	\usepackage{fontspec}
    \fi

    % Basic figure setup, for now with no caption control since it's done
    % automatically by Pandoc (which extracts ![](path) syntax from Markdown).
    \usepackage{graphicx}
    % Maintain compatibility with old templates. Remove in nbconvert 6.0
    \let\Oldincludegraphics\includegraphics
    % Ensure that by default, figures have no caption (until we provide a
    % proper Figure object with a Caption API and a way to capture that
    % in the conversion process - todo).
    \usepackage{caption}
    \DeclareCaptionFormat{nocaption}{}
    \captionsetup{format=nocaption,aboveskip=0pt,belowskip=0pt}

    \usepackage{float}
    \floatplacement{figure}{H} % forces figures to be placed at the correct location
    \usepackage{xcolor} % Allow colors to be defined
    \usepackage{enumerate} % Needed for markdown enumerations to work
    \usepackage{geometry} % Used to adjust the document margins
    \usepackage{amsmath} % Equations
    \usepackage{amssymb} % Equations
    \usepackage{textcomp} % defines textquotesingle
    % Hack from http://tex.stackexchange.com/a/47451/13684:
    \AtBeginDocument{%
        \def\PYZsq{\textquotesingle}% Upright quotes in Pygmentized code
    }
    \usepackage{upquote} % Upright quotes for verbatim code
    \usepackage{eurosym} % defines \euro
    \usepackage[mathletters]{ucs} % Extended unicode (utf-8) support
    \usepackage{fancyvrb} % verbatim replacement that allows latex
    \usepackage{grffile} % extends the file name processing of package graphics 
                         % to support a larger range
    \makeatletter % fix for old versions of grffile with XeLaTeX
    \@ifpackagelater{grffile}{2019/11/01}
    {
      % Do nothing on new versions
    }
    {
      \def\Gread@@xetex#1{%
        \IfFileExists{"\Gin@base".bb}%
        {\Gread@eps{\Gin@base.bb}}%
        {\Gread@@xetex@aux#1}%
      }
    }
    \makeatother
    \usepackage[Export]{adjustbox} % Used to constrain images to a maximum size
    \adjustboxset{max size={0.9\linewidth}{0.9\paperheight}}

    % The hyperref package gives us a pdf with properly built
    % internal navigation ('pdf bookmarks' for the table of contents,
    % internal cross-reference links, web links for URLs, etc.)
    \usepackage{hyperref}
    % The default LaTeX title has an obnoxious amount of whitespace. By default,
    % titling removes some of it. It also provides customization options.
    \usepackage{titling}
    \usepackage{longtable} % longtable support required by pandoc >1.10
    \usepackage{booktabs}  % table support for pandoc > 1.12.2
    \usepackage[inline]{enumitem} % IRkernel/repr support (it uses the enumerate* environment)
    \usepackage[normalem]{ulem} % ulem is needed to support strikethroughs (\sout)
                                % normalem makes italics be italics, not underlines
    \usepackage{mathrsfs}
    

    
    % Colors for the hyperref package
    \definecolor{urlcolor}{rgb}{0,.145,.698}
    \definecolor{linkcolor}{rgb}{.71,0.21,0.01}
    \definecolor{citecolor}{rgb}{.12,.54,.11}

    % ANSI colors
    \definecolor{ansi-black}{HTML}{3E424D}
    \definecolor{ansi-black-intense}{HTML}{282C36}
    \definecolor{ansi-red}{HTML}{E75C58}
    \definecolor{ansi-red-intense}{HTML}{B22B31}
    \definecolor{ansi-green}{HTML}{00A250}
    \definecolor{ansi-green-intense}{HTML}{007427}
    \definecolor{ansi-yellow}{HTML}{DDB62B}
    \definecolor{ansi-yellow-intense}{HTML}{B27D12}
    \definecolor{ansi-blue}{HTML}{208FFB}
    \definecolor{ansi-blue-intense}{HTML}{0065CA}
    \definecolor{ansi-magenta}{HTML}{D160C4}
    \definecolor{ansi-magenta-intense}{HTML}{A03196}
    \definecolor{ansi-cyan}{HTML}{60C6C8}
    \definecolor{ansi-cyan-intense}{HTML}{258F8F}
    \definecolor{ansi-white}{HTML}{C5C1B4}
    \definecolor{ansi-white-intense}{HTML}{A1A6B2}
    \definecolor{ansi-default-inverse-fg}{HTML}{FFFFFF}
    \definecolor{ansi-default-inverse-bg}{HTML}{000000}

    % common color for the border for error outputs.
    \definecolor{outerrorbackground}{HTML}{FFDFDF}

    % commands and environments needed by pandoc snippets
    % extracted from the output of `pandoc -s`
    \providecommand{\tightlist}{%
      \setlength{\itemsep}{0pt}\setlength{\parskip}{0pt}}
    \DefineVerbatimEnvironment{Highlighting}{Verbatim}{commandchars=\\\{\}}
    % Add ',fontsize=\small' for more characters per line
    \newenvironment{Shaded}{}{}
    \newcommand{\KeywordTok}[1]{\textcolor[rgb]{0.00,0.44,0.13}{\textbf{{#1}}}}
    \newcommand{\DataTypeTok}[1]{\textcolor[rgb]{0.56,0.13,0.00}{{#1}}}
    \newcommand{\DecValTok}[1]{\textcolor[rgb]{0.25,0.63,0.44}{{#1}}}
    \newcommand{\BaseNTok}[1]{\textcolor[rgb]{0.25,0.63,0.44}{{#1}}}
    \newcommand{\FloatTok}[1]{\textcolor[rgb]{0.25,0.63,0.44}{{#1}}}
    \newcommand{\CharTok}[1]{\textcolor[rgb]{0.25,0.44,0.63}{{#1}}}
    \newcommand{\StringTok}[1]{\textcolor[rgb]{0.25,0.44,0.63}{{#1}}}
    \newcommand{\CommentTok}[1]{\textcolor[rgb]{0.38,0.63,0.69}{\textit{{#1}}}}
    \newcommand{\OtherTok}[1]{\textcolor[rgb]{0.00,0.44,0.13}{{#1}}}
    \newcommand{\AlertTok}[1]{\textcolor[rgb]{1.00,0.00,0.00}{\textbf{{#1}}}}
    \newcommand{\FunctionTok}[1]{\textcolor[rgb]{0.02,0.16,0.49}{{#1}}}
    \newcommand{\RegionMarkerTok}[1]{{#1}}
    \newcommand{\ErrorTok}[1]{\textcolor[rgb]{1.00,0.00,0.00}{\textbf{{#1}}}}
    \newcommand{\NormalTok}[1]{{#1}}
    
    % Additional commands for more recent versions of Pandoc
    \newcommand{\ConstantTok}[1]{\textcolor[rgb]{0.53,0.00,0.00}{{#1}}}
    \newcommand{\SpecialCharTok}[1]{\textcolor[rgb]{0.25,0.44,0.63}{{#1}}}
    \newcommand{\VerbatimStringTok}[1]{\textcolor[rgb]{0.25,0.44,0.63}{{#1}}}
    \newcommand{\SpecialStringTok}[1]{\textcolor[rgb]{0.73,0.40,0.53}{{#1}}}
    \newcommand{\ImportTok}[1]{{#1}}
    \newcommand{\DocumentationTok}[1]{\textcolor[rgb]{0.73,0.13,0.13}{\textit{{#1}}}}
    \newcommand{\AnnotationTok}[1]{\textcolor[rgb]{0.38,0.63,0.69}{\textbf{\textit{{#1}}}}}
    \newcommand{\CommentVarTok}[1]{\textcolor[rgb]{0.38,0.63,0.69}{\textbf{\textit{{#1}}}}}
    \newcommand{\VariableTok}[1]{\textcolor[rgb]{0.10,0.09,0.49}{{#1}}}
    \newcommand{\ControlFlowTok}[1]{\textcolor[rgb]{0.00,0.44,0.13}{\textbf{{#1}}}}
    \newcommand{\OperatorTok}[1]{\textcolor[rgb]{0.40,0.40,0.40}{{#1}}}
    \newcommand{\BuiltInTok}[1]{{#1}}
    \newcommand{\ExtensionTok}[1]{{#1}}
    \newcommand{\PreprocessorTok}[1]{\textcolor[rgb]{0.74,0.48,0.00}{{#1}}}
    \newcommand{\AttributeTok}[1]{\textcolor[rgb]{0.49,0.56,0.16}{{#1}}}
    \newcommand{\InformationTok}[1]{\textcolor[rgb]{0.38,0.63,0.69}{\textbf{\textit{{#1}}}}}
    \newcommand{\WarningTok}[1]{\textcolor[rgb]{0.38,0.63,0.69}{\textbf{\textit{{#1}}}}}
    
    
    % Define a nice break command that doesn't care if a line doesn't already
    % exist.
    \def\br{\hspace*{\fill} \\* }
    % Math Jax compatibility definitions
    \def\gt{>}
    \def\lt{<}
    \let\Oldtex\TeX
    \let\Oldlatex\LaTeX
    \renewcommand{\TeX}{\textrm{\Oldtex}}
    \renewcommand{\LaTeX}{\textrm{\Oldlatex}}
    % Document parameters
    % Document title
    \title{IFT\_6758\_HW\_2\_(En)}
    
    
    
    
    
% Pygments definitions
\makeatletter
\def\PY@reset{\let\PY@it=\relax \let\PY@bf=\relax%
    \let\PY@ul=\relax \let\PY@tc=\relax%
    \let\PY@bc=\relax \let\PY@ff=\relax}
\def\PY@tok#1{\csname PY@tok@#1\endcsname}
\def\PY@toks#1+{\ifx\relax#1\empty\else%
    \PY@tok{#1}\expandafter\PY@toks\fi}
\def\PY@do#1{\PY@bc{\PY@tc{\PY@ul{%
    \PY@it{\PY@bf{\PY@ff{#1}}}}}}}
\def\PY#1#2{\PY@reset\PY@toks#1+\relax+\PY@do{#2}}

\expandafter\def\csname PY@tok@w\endcsname{\def\PY@tc##1{\textcolor[rgb]{0.73,0.73,0.73}{##1}}}
\expandafter\def\csname PY@tok@c\endcsname{\let\PY@it=\textit\def\PY@tc##1{\textcolor[rgb]{0.25,0.50,0.50}{##1}}}
\expandafter\def\csname PY@tok@cp\endcsname{\def\PY@tc##1{\textcolor[rgb]{0.74,0.48,0.00}{##1}}}
\expandafter\def\csname PY@tok@k\endcsname{\let\PY@bf=\textbf\def\PY@tc##1{\textcolor[rgb]{0.00,0.50,0.00}{##1}}}
\expandafter\def\csname PY@tok@kp\endcsname{\def\PY@tc##1{\textcolor[rgb]{0.00,0.50,0.00}{##1}}}
\expandafter\def\csname PY@tok@kt\endcsname{\def\PY@tc##1{\textcolor[rgb]{0.69,0.00,0.25}{##1}}}
\expandafter\def\csname PY@tok@o\endcsname{\def\PY@tc##1{\textcolor[rgb]{0.40,0.40,0.40}{##1}}}
\expandafter\def\csname PY@tok@ow\endcsname{\let\PY@bf=\textbf\def\PY@tc##1{\textcolor[rgb]{0.67,0.13,1.00}{##1}}}
\expandafter\def\csname PY@tok@nb\endcsname{\def\PY@tc##1{\textcolor[rgb]{0.00,0.50,0.00}{##1}}}
\expandafter\def\csname PY@tok@nf\endcsname{\def\PY@tc##1{\textcolor[rgb]{0.00,0.00,1.00}{##1}}}
\expandafter\def\csname PY@tok@nc\endcsname{\let\PY@bf=\textbf\def\PY@tc##1{\textcolor[rgb]{0.00,0.00,1.00}{##1}}}
\expandafter\def\csname PY@tok@nn\endcsname{\let\PY@bf=\textbf\def\PY@tc##1{\textcolor[rgb]{0.00,0.00,1.00}{##1}}}
\expandafter\def\csname PY@tok@ne\endcsname{\let\PY@bf=\textbf\def\PY@tc##1{\textcolor[rgb]{0.82,0.25,0.23}{##1}}}
\expandafter\def\csname PY@tok@nv\endcsname{\def\PY@tc##1{\textcolor[rgb]{0.10,0.09,0.49}{##1}}}
\expandafter\def\csname PY@tok@no\endcsname{\def\PY@tc##1{\textcolor[rgb]{0.53,0.00,0.00}{##1}}}
\expandafter\def\csname PY@tok@nl\endcsname{\def\PY@tc##1{\textcolor[rgb]{0.63,0.63,0.00}{##1}}}
\expandafter\def\csname PY@tok@ni\endcsname{\let\PY@bf=\textbf\def\PY@tc##1{\textcolor[rgb]{0.60,0.60,0.60}{##1}}}
\expandafter\def\csname PY@tok@na\endcsname{\def\PY@tc##1{\textcolor[rgb]{0.49,0.56,0.16}{##1}}}
\expandafter\def\csname PY@tok@nt\endcsname{\let\PY@bf=\textbf\def\PY@tc##1{\textcolor[rgb]{0.00,0.50,0.00}{##1}}}
\expandafter\def\csname PY@tok@nd\endcsname{\def\PY@tc##1{\textcolor[rgb]{0.67,0.13,1.00}{##1}}}
\expandafter\def\csname PY@tok@s\endcsname{\def\PY@tc##1{\textcolor[rgb]{0.73,0.13,0.13}{##1}}}
\expandafter\def\csname PY@tok@sd\endcsname{\let\PY@it=\textit\def\PY@tc##1{\textcolor[rgb]{0.73,0.13,0.13}{##1}}}
\expandafter\def\csname PY@tok@si\endcsname{\let\PY@bf=\textbf\def\PY@tc##1{\textcolor[rgb]{0.73,0.40,0.53}{##1}}}
\expandafter\def\csname PY@tok@se\endcsname{\let\PY@bf=\textbf\def\PY@tc##1{\textcolor[rgb]{0.73,0.40,0.13}{##1}}}
\expandafter\def\csname PY@tok@sr\endcsname{\def\PY@tc##1{\textcolor[rgb]{0.73,0.40,0.53}{##1}}}
\expandafter\def\csname PY@tok@ss\endcsname{\def\PY@tc##1{\textcolor[rgb]{0.10,0.09,0.49}{##1}}}
\expandafter\def\csname PY@tok@sx\endcsname{\def\PY@tc##1{\textcolor[rgb]{0.00,0.50,0.00}{##1}}}
\expandafter\def\csname PY@tok@m\endcsname{\def\PY@tc##1{\textcolor[rgb]{0.40,0.40,0.40}{##1}}}
\expandafter\def\csname PY@tok@gh\endcsname{\let\PY@bf=\textbf\def\PY@tc##1{\textcolor[rgb]{0.00,0.00,0.50}{##1}}}
\expandafter\def\csname PY@tok@gu\endcsname{\let\PY@bf=\textbf\def\PY@tc##1{\textcolor[rgb]{0.50,0.00,0.50}{##1}}}
\expandafter\def\csname PY@tok@gd\endcsname{\def\PY@tc##1{\textcolor[rgb]{0.63,0.00,0.00}{##1}}}
\expandafter\def\csname PY@tok@gi\endcsname{\def\PY@tc##1{\textcolor[rgb]{0.00,0.63,0.00}{##1}}}
\expandafter\def\csname PY@tok@gr\endcsname{\def\PY@tc##1{\textcolor[rgb]{1.00,0.00,0.00}{##1}}}
\expandafter\def\csname PY@tok@ge\endcsname{\let\PY@it=\textit}
\expandafter\def\csname PY@tok@gs\endcsname{\let\PY@bf=\textbf}
\expandafter\def\csname PY@tok@gp\endcsname{\let\PY@bf=\textbf\def\PY@tc##1{\textcolor[rgb]{0.00,0.00,0.50}{##1}}}
\expandafter\def\csname PY@tok@go\endcsname{\def\PY@tc##1{\textcolor[rgb]{0.53,0.53,0.53}{##1}}}
\expandafter\def\csname PY@tok@gt\endcsname{\def\PY@tc##1{\textcolor[rgb]{0.00,0.27,0.87}{##1}}}
\expandafter\def\csname PY@tok@err\endcsname{\def\PY@bc##1{\setlength{\fboxsep}{0pt}\fcolorbox[rgb]{1.00,0.00,0.00}{1,1,1}{\strut ##1}}}
\expandafter\def\csname PY@tok@kc\endcsname{\let\PY@bf=\textbf\def\PY@tc##1{\textcolor[rgb]{0.00,0.50,0.00}{##1}}}
\expandafter\def\csname PY@tok@kd\endcsname{\let\PY@bf=\textbf\def\PY@tc##1{\textcolor[rgb]{0.00,0.50,0.00}{##1}}}
\expandafter\def\csname PY@tok@kn\endcsname{\let\PY@bf=\textbf\def\PY@tc##1{\textcolor[rgb]{0.00,0.50,0.00}{##1}}}
\expandafter\def\csname PY@tok@kr\endcsname{\let\PY@bf=\textbf\def\PY@tc##1{\textcolor[rgb]{0.00,0.50,0.00}{##1}}}
\expandafter\def\csname PY@tok@bp\endcsname{\def\PY@tc##1{\textcolor[rgb]{0.00,0.50,0.00}{##1}}}
\expandafter\def\csname PY@tok@fm\endcsname{\def\PY@tc##1{\textcolor[rgb]{0.00,0.00,1.00}{##1}}}
\expandafter\def\csname PY@tok@vc\endcsname{\def\PY@tc##1{\textcolor[rgb]{0.10,0.09,0.49}{##1}}}
\expandafter\def\csname PY@tok@vg\endcsname{\def\PY@tc##1{\textcolor[rgb]{0.10,0.09,0.49}{##1}}}
\expandafter\def\csname PY@tok@vi\endcsname{\def\PY@tc##1{\textcolor[rgb]{0.10,0.09,0.49}{##1}}}
\expandafter\def\csname PY@tok@vm\endcsname{\def\PY@tc##1{\textcolor[rgb]{0.10,0.09,0.49}{##1}}}
\expandafter\def\csname PY@tok@sa\endcsname{\def\PY@tc##1{\textcolor[rgb]{0.73,0.13,0.13}{##1}}}
\expandafter\def\csname PY@tok@sb\endcsname{\def\PY@tc##1{\textcolor[rgb]{0.73,0.13,0.13}{##1}}}
\expandafter\def\csname PY@tok@sc\endcsname{\def\PY@tc##1{\textcolor[rgb]{0.73,0.13,0.13}{##1}}}
\expandafter\def\csname PY@tok@dl\endcsname{\def\PY@tc##1{\textcolor[rgb]{0.73,0.13,0.13}{##1}}}
\expandafter\def\csname PY@tok@s2\endcsname{\def\PY@tc##1{\textcolor[rgb]{0.73,0.13,0.13}{##1}}}
\expandafter\def\csname PY@tok@sh\endcsname{\def\PY@tc##1{\textcolor[rgb]{0.73,0.13,0.13}{##1}}}
\expandafter\def\csname PY@tok@s1\endcsname{\def\PY@tc##1{\textcolor[rgb]{0.73,0.13,0.13}{##1}}}
\expandafter\def\csname PY@tok@mb\endcsname{\def\PY@tc##1{\textcolor[rgb]{0.40,0.40,0.40}{##1}}}
\expandafter\def\csname PY@tok@mf\endcsname{\def\PY@tc##1{\textcolor[rgb]{0.40,0.40,0.40}{##1}}}
\expandafter\def\csname PY@tok@mh\endcsname{\def\PY@tc##1{\textcolor[rgb]{0.40,0.40,0.40}{##1}}}
\expandafter\def\csname PY@tok@mi\endcsname{\def\PY@tc##1{\textcolor[rgb]{0.40,0.40,0.40}{##1}}}
\expandafter\def\csname PY@tok@il\endcsname{\def\PY@tc##1{\textcolor[rgb]{0.40,0.40,0.40}{##1}}}
\expandafter\def\csname PY@tok@mo\endcsname{\def\PY@tc##1{\textcolor[rgb]{0.40,0.40,0.40}{##1}}}
\expandafter\def\csname PY@tok@ch\endcsname{\let\PY@it=\textit\def\PY@tc##1{\textcolor[rgb]{0.25,0.50,0.50}{##1}}}
\expandafter\def\csname PY@tok@cm\endcsname{\let\PY@it=\textit\def\PY@tc##1{\textcolor[rgb]{0.25,0.50,0.50}{##1}}}
\expandafter\def\csname PY@tok@cpf\endcsname{\let\PY@it=\textit\def\PY@tc##1{\textcolor[rgb]{0.25,0.50,0.50}{##1}}}
\expandafter\def\csname PY@tok@c1\endcsname{\let\PY@it=\textit\def\PY@tc##1{\textcolor[rgb]{0.25,0.50,0.50}{##1}}}
\expandafter\def\csname PY@tok@cs\endcsname{\let\PY@it=\textit\def\PY@tc##1{\textcolor[rgb]{0.25,0.50,0.50}{##1}}}

\def\PYZbs{\char`\\}
\def\PYZus{\char`\_}
\def\PYZob{\char`\{}
\def\PYZcb{\char`\}}
\def\PYZca{\char`\^}
\def\PYZam{\char`\&}
\def\PYZlt{\char`\<}
\def\PYZgt{\char`\>}
\def\PYZsh{\char`\#}
\def\PYZpc{\char`\%}
\def\PYZdl{\char`\$}
\def\PYZhy{\char`\-}
\def\PYZsq{\char`\'}
\def\PYZdq{\char`\"}
\def\PYZti{\char`\~}
% for compatibility with earlier versions
\def\PYZat{@}
\def\PYZlb{[}
\def\PYZrb{]}
\makeatother


    % For linebreaks inside Verbatim environment from package fancyvrb. 
    \makeatletter
        \newbox\Wrappedcontinuationbox 
        \newbox\Wrappedvisiblespacebox 
        \newcommand*\Wrappedvisiblespace {\textcolor{red}{\textvisiblespace}} 
        \newcommand*\Wrappedcontinuationsymbol {\textcolor{red}{\llap{\tiny$\m@th\hookrightarrow$}}} 
        \newcommand*\Wrappedcontinuationindent {3ex } 
        \newcommand*\Wrappedafterbreak {\kern\Wrappedcontinuationindent\copy\Wrappedcontinuationbox} 
        % Take advantage of the already applied Pygments mark-up to insert 
        % potential linebreaks for TeX processing. 
        %        {, <, #, %, $, ' and ": go to next line. 
        %        _, }, ^, &, >, - and ~: stay at end of broken line. 
        % Use of \textquotesingle for straight quote. 
        \newcommand*\Wrappedbreaksatspecials {% 
            \def\PYGZus{\discretionary{\char`\_}{\Wrappedafterbreak}{\char`\_}}% 
            \def\PYGZob{\discretionary{}{\Wrappedafterbreak\char`\{}{\char`\{}}% 
            \def\PYGZcb{\discretionary{\char`\}}{\Wrappedafterbreak}{\char`\}}}% 
            \def\PYGZca{\discretionary{\char`\^}{\Wrappedafterbreak}{\char`\^}}% 
            \def\PYGZam{\discretionary{\char`\&}{\Wrappedafterbreak}{\char`\&}}% 
            \def\PYGZlt{\discretionary{}{\Wrappedafterbreak\char`\<}{\char`\<}}% 
            \def\PYGZgt{\discretionary{\char`\>}{\Wrappedafterbreak}{\char`\>}}% 
            \def\PYGZsh{\discretionary{}{\Wrappedafterbreak\char`\#}{\char`\#}}% 
            \def\PYGZpc{\discretionary{}{\Wrappedafterbreak\char`\%}{\char`\%}}% 
            \def\PYGZdl{\discretionary{}{\Wrappedafterbreak\char`\$}{\char`\$}}% 
            \def\PYGZhy{\discretionary{\char`\-}{\Wrappedafterbreak}{\char`\-}}% 
            \def\PYGZsq{\discretionary{}{\Wrappedafterbreak\textquotesingle}{\textquotesingle}}% 
            \def\PYGZdq{\discretionary{}{\Wrappedafterbreak\char`\"}{\char`\"}}% 
            \def\PYGZti{\discretionary{\char`\~}{\Wrappedafterbreak}{\char`\~}}% 
        } 
        % Some characters . , ; ? ! / are not pygmentized. 
        % This macro makes them "active" and they will insert potential linebreaks 
        \newcommand*\Wrappedbreaksatpunct {% 
            \lccode`\~`\.\lowercase{\def~}{\discretionary{\hbox{\char`\.}}{\Wrappedafterbreak}{\hbox{\char`\.}}}% 
            \lccode`\~`\,\lowercase{\def~}{\discretionary{\hbox{\char`\,}}{\Wrappedafterbreak}{\hbox{\char`\,}}}% 
            \lccode`\~`\;\lowercase{\def~}{\discretionary{\hbox{\char`\;}}{\Wrappedafterbreak}{\hbox{\char`\;}}}% 
            \lccode`\~`\:\lowercase{\def~}{\discretionary{\hbox{\char`\:}}{\Wrappedafterbreak}{\hbox{\char`\:}}}% 
            \lccode`\~`\?\lowercase{\def~}{\discretionary{\hbox{\char`\?}}{\Wrappedafterbreak}{\hbox{\char`\?}}}% 
            \lccode`\~`\!\lowercase{\def~}{\discretionary{\hbox{\char`\!}}{\Wrappedafterbreak}{\hbox{\char`\!}}}% 
            \lccode`\~`\/\lowercase{\def~}{\discretionary{\hbox{\char`\/}}{\Wrappedafterbreak}{\hbox{\char`\/}}}% 
            \catcode`\.\active
            \catcode`\,\active 
            \catcode`\;\active
            \catcode`\:\active
            \catcode`\?\active
            \catcode`\!\active
            \catcode`\/\active 
            \lccode`\~`\~ 	
        }
    \makeatother

    \let\OriginalVerbatim=\Verbatim
    \makeatletter
    \renewcommand{\Verbatim}[1][1]{%
        %\parskip\z@skip
        \sbox\Wrappedcontinuationbox {\Wrappedcontinuationsymbol}%
        \sbox\Wrappedvisiblespacebox {\FV@SetupFont\Wrappedvisiblespace}%
        \def\FancyVerbFormatLine ##1{\hsize\linewidth
            \vtop{\raggedright\hyphenpenalty\z@\exhyphenpenalty\z@
                \doublehyphendemerits\z@\finalhyphendemerits\z@
                \strut ##1\strut}%
        }%
        % If the linebreak is at a space, the latter will be displayed as visible
        % space at end of first line, and a continuation symbol starts next line.
        % Stretch/shrink are however usually zero for typewriter font.
        \def\FV@Space {%
            \nobreak\hskip\z@ plus\fontdimen3\font minus\fontdimen4\font
            \discretionary{\copy\Wrappedvisiblespacebox}{\Wrappedafterbreak}
            {\kern\fontdimen2\font}%
        }%
        
        % Allow breaks at special characters using \PYG... macros.
        \Wrappedbreaksatspecials
        % Breaks at punctuation characters . , ; ? ! and / need catcode=\active 	
        \OriginalVerbatim[#1,codes*=\Wrappedbreaksatpunct]%
    }
    \makeatother

    % Exact colors from NB
    \definecolor{incolor}{HTML}{303F9F}
    \definecolor{outcolor}{HTML}{D84315}
    \definecolor{cellborder}{HTML}{CFCFCF}
    \definecolor{cellbackground}{HTML}{F7F7F7}
    
    % prompt
    \makeatletter
    \newcommand{\boxspacing}{\kern\kvtcb@left@rule\kern\kvtcb@boxsep}
    \makeatother
    \newcommand{\prompt}[4]{
        {\ttfamily\llap{{\color{#2}[#3]:\hspace{3pt}#4}}\vspace{-\baselineskip}}
    }
    

    
    % Prevent overflowing lines due to hard-to-break entities
    \sloppy 
    % Setup hyperref package
    \hypersetup{
      breaklinks=true,  % so long urls are correctly broken across lines
      colorlinks=true,
      urlcolor=urlcolor,
      linkcolor=linkcolor,
      citecolor=citecolor,
      }
    % Slightly bigger margins than the latex defaults
    
    \geometry{verbose,tmargin=1in,bmargin=1in,lmargin=1in,rmargin=1in}
    
    

\begin{document}
    
    \maketitle
    
    

    
    IFT-6758 : Data Science

Fall - 2020

Homework - 2 - Matteo Esposito

    \href{https://colab.research.google.com/drive/1CKUlvdEa1bJLS2_P7UeaPUWSZ_ZDUDSH}{Notebook}
due November 06, 2020 at
\href{https://www.worldtimebuddy.com/?qm=1\&lid=6077243\&h=6077243\&date=2020-11-06\&sln=23-24}{23.59
EST} as \textbf{PDF} on
\href{https://www.gradescope.com/courses/179325/assignments/773268}{Gradescope}

    \begin{tcolorbox}[breakable, size=fbox, boxrule=1pt, pad at break*=1mm,colback=cellbackground, colframe=cellborder]
\prompt{In}{incolor}{1}{\boxspacing}
\begin{Verbatim}[commandchars=\\\{\}]
\PY{c+c1}{\PYZsh{}@title Imports (Run this cell first) \PYZob{} run: \PYZdq{}auto\PYZdq{} \PYZcb{}}
\PY{n}{plotting\PYZus{}library} \PY{o}{=} \PY{l+s+s2}{\PYZdq{}}\PY{l+s+s2}{matplotlib}\PY{l+s+s2}{\PYZdq{}}
\PY{k+kn}{import} \PY{n+nn}{pandas} \PY{k}{as} \PY{n+nn}{pd}
\PY{k+kn}{import} \PY{n+nn}{numpy} \PY{k}{as} \PY{n+nn}{np}

\PY{k+kn}{from} \PY{n+nn}{sklearn}\PY{n+nn}{.}\PY{n+nn}{decomposition} \PY{k+kn}{import} \PY{n}{PCA}
\PY{k+kn}{from} \PY{n+nn}{sklearn}\PY{n+nn}{.}\PY{n+nn}{linear\PYZus{}model} \PY{k+kn}{import} \PY{n}{LinearRegression}

\PY{c+c1}{\PYZsh{} Not mandatory to use}
\PY{k+kn}{from} \PY{n+nn}{sklearn}\PY{n+nn}{.}\PY{n+nn}{utils} \PY{k+kn}{import} \PY{n}{resample}

\PY{k+kn}{import} \PY{n+nn}{matplotlib}\PY{n+nn}{.}\PY{n+nn}{pyplot} \PY{k}{as} \PY{n+nn}{plt}

\PY{k+kn}{import} \PY{n+nn}{seaborn} \PY{k}{as} \PY{n+nn}{sns}

\PY{c+c1}{\PYZsh{} Uncomment this line below if using seaborn}
\PY{n}{sns}\PY{o}{.}\PY{n}{set}\PY{p}{(}\PY{p}{)} 

\PY{o}{\PYZpc{}}\PY{k}{matplotlib} inline

\PY{n}{path} \PY{o}{=} \PY{l+s+s1}{\PYZsq{}}\PY{l+s+s1}{https://raw.githubusercontent.com/Jhelum\PYZhy{}Ch/DataScience\PYZus{}IFT6758/gh\PYZhy{}pages/media/}\PY{l+s+si}{\PYZob{}\PYZcb{}}\PY{l+s+s1}{\PYZsq{}}
\end{Verbatim}
\end{tcolorbox}

    \hypertarget{pca}{%
\subsection{PCA}\label{pca}}

    \hypertarget{q1}{%
\paragraph{\texorpdfstring{\textbf{Q1}}{Q1}}\label{q1}}

    \textbf{12 points} = \((1.5 + 2 + 2 + 1.5 + 1.5 + 2 + 1.5)\)

    The cell below loads a subset of the California Housing dataset.

\begin{enumerate}
\def\labelenumi{(\alph{enumi})}
\item
  Store only the \texttt{latitude}, \texttt{longitude} and
  \texttt{median\_house\_value} columns in a dataframe denoted by a
  variable \texttt{features}. Produce a scatter plot of the data points
  with \texttt{longitude} along x-axis, \texttt{latitude} along y-axis
  and the points colored by \texttt{median\_house\_value} i.e.~higher
  the \texttt{median\_house\_value}, darker the data point in the plot.
\item
  Perform a PCA on the subset of the dataframe you created in (a) with
  only the \texttt{latitude} and \texttt{longitude} columns. Produce a
  scatter plot of the transformed data points with the first principal
  component \texttt{PC\ 1} along x-axis and second principal component
  \texttt{PC\ 2} along y-axis and the points colored by
  \texttt{median\_house\_value} just like in (a).
\item
  Provide a simple interpretation for what the first principal component
  \texttt{PC\ 1} could possibly represent in the plot in (b) by
  comparing it with that in (a). \textbf{Justify} your answer.
\item
  Repeat what you did in (b) above by setting the \texttt{whiten}
  parameter as \texttt{True} in the \texttt{PCA()} constructor and
  producing the plot. What difference do you observe? What do you think
  \texttt{whiten} does specifically in this problem?
\item
  Perform a PCA on the entire dataframe \texttt{features} with
  \texttt{whiten} set to \texttt{True} and produce a scatter plot of the
  transformed data points with the first principal component
  \texttt{PC\ 1} along x-axis and second principal component
  \texttt{PC\ 2} along y-axis and the points colored by
  \texttt{median\_house\_value} just like in (a).
\item
  Observe how the color indicating \texttt{median\_house\_value} varies
  in the plot you produced in (e). Is the variation of
  \texttt{median\_house\_value} depicted in this plot simpler than what
  is indicated by all the above plots? Provide an \textbf{explanation}
  for why it is (or) it is not the case.
\item
  The California Department of Housing and Community Development (HCD)
  releases additional information about the data samples you used here,
  by providing an \texttt{price\_index} tag that can take values
  \texttt{high}, \texttt{middle} or \texttt{low} based
  \texttt{median\_house\_value}. If you were to eventually use the
  principal components you produced in (e), which one(s) among the
  \texttt{PC\ 1}, \texttt{PC\ 2}, ..,. would you use to classify the
  data samples into these three categories (\texttt{high},
  \texttt{middle} and \texttt{low})? Concretely \textbf{justify} your
  choice.
\end{enumerate}

\textbf{Bonus : (3 points)} \textgreater{} (h) In the plot that you
observe in (a), you will remark two major clusters that are the
\emph{darkest}. Let us identify the cluster with the higher value of
\texttt{latitude} as the \texttt{SF\ cluster} and the one with the lower
value of \texttt{latitude} as the \texttt{LA\ cluster}. Verify
programmatically if this clustering is preserved or distorted in the
plot in (e). What does this tell you about what is represented by the
second principal component \texttt{PC\ 2} produced in (e)?

    \begin{tcolorbox}[breakable, size=fbox, boxrule=1pt, pad at break*=1mm,colback=cellbackground, colframe=cellborder]
\prompt{In}{incolor}{2}{\boxspacing}
\begin{Verbatim}[commandchars=\\\{\}]
\PY{n}{housing} \PY{o}{=} \PY{n}{pd}\PY{o}{.}\PY{n}{read\PYZus{}csv}\PY{p}{(}\PY{n}{path}\PY{o}{.}\PY{n}{format}\PY{p}{(}\PY{l+s+s1}{\PYZsq{}}\PY{l+s+s1}{california\PYZus{}housing.csv}\PY{l+s+s1}{\PYZsq{}}\PY{p}{)}\PY{p}{)}
\end{Verbatim}
\end{tcolorbox}

    \begin{enumerate}
\def\labelenumi{(\alph{enumi})}
\tightlist
\item
\end{enumerate}

    \begin{tcolorbox}[breakable, size=fbox, boxrule=1pt, pad at break*=1mm,colback=cellbackground, colframe=cellborder]
\prompt{In}{incolor}{133}{\boxspacing}
\begin{Verbatim}[commandchars=\\\{\}]
\PY{n}{features} \PY{o}{=} \PY{n}{housing}\PY{p}{[}\PY{p}{[}\PY{l+s+s1}{\PYZsq{}}\PY{l+s+s1}{latitude}\PY{l+s+s1}{\PYZsq{}}\PY{p}{,} \PY{l+s+s1}{\PYZsq{}}\PY{l+s+s1}{longitude}\PY{l+s+s1}{\PYZsq{}}\PY{p}{,} \PY{l+s+s1}{\PYZsq{}}\PY{l+s+s1}{median\PYZus{}house\PYZus{}value}\PY{l+s+s1}{\PYZsq{}}\PY{p}{]}\PY{p}{]}
\PY{n}{features}\PY{o}{.}\PY{n}{head}\PY{p}{(}\PY{p}{)}
\end{Verbatim}
\end{tcolorbox}

            \begin{tcolorbox}[breakable, size=fbox, boxrule=.5pt, pad at break*=1mm, opacityfill=0]
\prompt{Out}{outcolor}{133}{\boxspacing}
\begin{Verbatim}[commandchars=\\\{\}]
   latitude  longitude  median\_house\_value
0     34.19    -114.31             66900.0
1     34.40    -114.47             80100.0
2     33.69    -114.56             85700.0
3     33.64    -114.57             73400.0
4     33.57    -114.57             65500.0
\end{Verbatim}
\end{tcolorbox}
        
    \begin{tcolorbox}[breakable, size=fbox, boxrule=1pt, pad at break*=1mm,colback=cellbackground, colframe=cellborder]
\prompt{In}{incolor}{134}{\boxspacing}
\begin{Verbatim}[commandchars=\\\{\}]
\PY{n}{plt}\PY{o}{.}\PY{n}{figure}\PY{p}{(}\PY{n}{figsize}\PY{o}{=}\PY{p}{(}\PY{l+m+mi}{12}\PY{p}{,}\PY{l+m+mi}{7}\PY{p}{)}\PY{p}{)}
\PY{n}{sns}\PY{o}{.}\PY{n}{scatterplot}\PY{p}{(}\PY{n}{data}\PY{o}{=}\PY{n}{features}\PY{p}{,} \PY{n}{x}\PY{o}{=}\PY{l+s+s1}{\PYZsq{}}\PY{l+s+s1}{longitude}\PY{l+s+s1}{\PYZsq{}}\PY{p}{,} \PY{n}{y}\PY{o}{=}\PY{l+s+s1}{\PYZsq{}}\PY{l+s+s1}{latitude}\PY{l+s+s1}{\PYZsq{}}\PY{p}{,} \PY{n}{hue}\PY{o}{=}\PY{l+s+s1}{\PYZsq{}}\PY{l+s+s1}{median\PYZus{}house\PYZus{}value}\PY{l+s+s1}{\PYZsq{}}\PY{p}{)}\PY{p}{;}
\end{Verbatim}
\end{tcolorbox}

    \begin{center}
    \adjustimage{max size={0.9\linewidth}{0.9\paperheight}}{output_10_0.png}
    \end{center}
    { \hspace*{\fill} \\}
    
    \begin{enumerate}
\def\labelenumi{(\alph{enumi})}
\setcounter{enumi}{1}
\tightlist
\item
\end{enumerate}

    \begin{tcolorbox}[breakable, size=fbox, boxrule=1pt, pad at break*=1mm,colback=cellbackground, colframe=cellborder]
\prompt{In}{incolor}{138}{\boxspacing}
\begin{Verbatim}[commandchars=\\\{\}]
\PY{n}{pca} \PY{o}{=} \PY{n}{PCA}\PY{p}{(}\PY{n}{n\PYZus{}components}\PY{o}{=}\PY{l+m+mi}{2}\PY{p}{)}
\PY{n}{pcs} \PY{o}{=} \PY{n}{pca}\PY{o}{.}\PY{n}{fit\PYZus{}transform}\PY{p}{(}\PY{n}{features}\PY{p}{[}\PY{p}{[}\PY{l+s+s1}{\PYZsq{}}\PY{l+s+s1}{latitude}\PY{l+s+s1}{\PYZsq{}}\PY{p}{,} \PY{l+s+s1}{\PYZsq{}}\PY{l+s+s1}{longitude}\PY{l+s+s1}{\PYZsq{}}\PY{p}{]}\PY{p}{]}\PY{p}{)}
\PY{n}{pca\PYZus{}df} \PY{o}{=} \PY{n}{pd}\PY{o}{.}\PY{n}{DataFrame}\PY{p}{(}\PY{n}{data}\PY{o}{=}\PY{n}{pcs}\PY{p}{,} \PY{n}{columns} \PY{o}{=} \PY{p}{[}\PY{l+s+s1}{\PYZsq{}}\PY{l+s+s1}{PC 1}\PY{l+s+s1}{\PYZsq{}}\PY{p}{,} \PY{l+s+s1}{\PYZsq{}}\PY{l+s+s1}{PC 2}\PY{l+s+s1}{\PYZsq{}}\PY{p}{]}\PY{p}{)}
\PY{n}{pca\PYZus{}df}\PY{p}{[}\PY{l+s+s1}{\PYZsq{}}\PY{l+s+s1}{median\PYZus{}house\PYZus{}value}\PY{l+s+s1}{\PYZsq{}}\PY{p}{]} \PY{o}{=} \PY{n}{features}\PY{p}{[}\PY{l+s+s1}{\PYZsq{}}\PY{l+s+s1}{median\PYZus{}house\PYZus{}value}\PY{l+s+s1}{\PYZsq{}}\PY{p}{]}
\end{Verbatim}
\end{tcolorbox}

    \begin{tcolorbox}[breakable, size=fbox, boxrule=1pt, pad at break*=1mm,colback=cellbackground, colframe=cellborder]
\prompt{In}{incolor}{139}{\boxspacing}
\begin{Verbatim}[commandchars=\\\{\}]
\PY{n}{pca\PYZus{}df}\PY{o}{.}\PY{n}{head}\PY{p}{(}\PY{p}{)}
\end{Verbatim}
\end{tcolorbox}

            \begin{tcolorbox}[breakable, size=fbox, boxrule=.5pt, pad at break*=1mm, opacityfill=0]
\prompt{Out}{outcolor}{139}{\boxspacing}
\begin{Verbatim}[commandchars=\\\{\}]
       PC 1      PC 2  median\_house\_value
0 -4.632846  2.860289             66900.0
1 -4.370153  2.886607             80100.0
2 -4.827792  2.336365             85700.0
3 -4.857522  2.294939             73400.0
4 -4.908696  2.247176             65500.0
\end{Verbatim}
\end{tcolorbox}
        
    \begin{tcolorbox}[breakable, size=fbox, boxrule=1pt, pad at break*=1mm,colback=cellbackground, colframe=cellborder]
\prompt{In}{incolor}{140}{\boxspacing}
\begin{Verbatim}[commandchars=\\\{\}]
\PY{n}{plt}\PY{o}{.}\PY{n}{figure}\PY{p}{(}\PY{n}{figsize}\PY{o}{=}\PY{p}{(}\PY{l+m+mi}{12}\PY{p}{,}\PY{l+m+mi}{7}\PY{p}{)}\PY{p}{)}
\PY{n}{sns}\PY{o}{.}\PY{n}{scatterplot}\PY{p}{(}\PY{n}{data}\PY{o}{=}\PY{n}{pca\PYZus{}df}\PY{p}{,} \PY{n}{x}\PY{o}{=}\PY{l+s+s1}{\PYZsq{}}\PY{l+s+s1}{PC 1}\PY{l+s+s1}{\PYZsq{}}\PY{p}{,} \PY{n}{y}\PY{o}{=}\PY{l+s+s1}{\PYZsq{}}\PY{l+s+s1}{PC 2}\PY{l+s+s1}{\PYZsq{}}\PY{p}{,} \PY{n}{hue}\PY{o}{=}\PY{l+s+s2}{\PYZdq{}}\PY{l+s+s2}{median\PYZus{}house\PYZus{}value}\PY{l+s+s2}{\PYZdq{}}\PY{p}{)}\PY{p}{;}
\end{Verbatim}
\end{tcolorbox}

    \begin{center}
    \adjustimage{max size={0.9\linewidth}{0.9\paperheight}}{output_14_0.png}
    \end{center}
    { \hspace*{\fill} \\}
    
    \begin{enumerate}
\def\labelenumi{(\alph{enumi})}
\setcounter{enumi}{2}
\tightlist
\item
\end{enumerate}

    PC1 likely represents the downward slope in the original dataset, from
top left to bottom right. In other words, it represents the decrease in
latitude and increase in longitude (this can be translated to the
south-western-ness of a point).

    \begin{enumerate}
\def\labelenumi{(\alph{enumi})}
\setcounter{enumi}{3}
\tightlist
\item
\end{enumerate}

    \begin{tcolorbox}[breakable, size=fbox, boxrule=1pt, pad at break*=1mm,colback=cellbackground, colframe=cellborder]
\prompt{In}{incolor}{8}{\boxspacing}
\begin{Verbatim}[commandchars=\\\{\}]
\PY{n}{pca} \PY{o}{=} \PY{n}{PCA}\PY{p}{(}\PY{n}{n\PYZus{}components}\PY{o}{=}\PY{l+m+mi}{2}\PY{p}{,} \PY{n}{whiten}\PY{o}{=}\PY{k+kc}{True}\PY{p}{)}
\PY{n}{pcs} \PY{o}{=} \PY{n}{pca}\PY{o}{.}\PY{n}{fit\PYZus{}transform}\PY{p}{(}\PY{n}{features}\PY{p}{[}\PY{p}{[}\PY{l+s+s1}{\PYZsq{}}\PY{l+s+s1}{latitude}\PY{l+s+s1}{\PYZsq{}}\PY{p}{,} \PY{l+s+s1}{\PYZsq{}}\PY{l+s+s1}{longitude}\PY{l+s+s1}{\PYZsq{}}\PY{p}{]}\PY{p}{]}\PY{p}{)}
\PY{n}{pca\PYZus{}df} \PY{o}{=} \PY{n}{pd}\PY{o}{.}\PY{n}{DataFrame}\PY{p}{(}\PY{n}{data}\PY{o}{=}\PY{n}{pcs}\PY{p}{,} \PY{n}{columns} \PY{o}{=} \PY{p}{[}\PY{l+s+s1}{\PYZsq{}}\PY{l+s+s1}{PC 1}\PY{l+s+s1}{\PYZsq{}}\PY{p}{,} \PY{l+s+s1}{\PYZsq{}}\PY{l+s+s1}{PC 2}\PY{l+s+s1}{\PYZsq{}}\PY{p}{]}\PY{p}{)}
\PY{n}{pca\PYZus{}df}\PY{p}{[}\PY{l+s+s1}{\PYZsq{}}\PY{l+s+s1}{median\PYZus{}house\PYZus{}value}\PY{l+s+s1}{\PYZsq{}}\PY{p}{]} \PY{o}{=} \PY{n}{features}\PY{p}{[}\PY{l+s+s1}{\PYZsq{}}\PY{l+s+s1}{median\PYZus{}house\PYZus{}value}\PY{l+s+s1}{\PYZsq{}}\PY{p}{]}
\end{Verbatim}
\end{tcolorbox}

    \begin{tcolorbox}[breakable, size=fbox, boxrule=1pt, pad at break*=1mm,colback=cellbackground, colframe=cellborder]
\prompt{In}{incolor}{9}{\boxspacing}
\begin{Verbatim}[commandchars=\\\{\}]
\PY{n}{pca\PYZus{}df}\PY{o}{.}\PY{n}{head}\PY{p}{(}\PY{p}{)}
\end{Verbatim}
\end{tcolorbox}

            \begin{tcolorbox}[breakable, size=fbox, boxrule=.5pt, pad at break*=1mm, opacityfill=0]
\prompt{Out}{outcolor}{9}{\boxspacing}
\begin{Verbatim}[commandchars=\\\{\}]
       PC 1      PC 2  median\_house\_value
0 -1.611088  5.057651             66900.0
1 -1.519736  5.104189             80100.0
2 -1.678881  4.131233             85700.0
3 -1.689220  4.057982             73400.0
4 -1.707015  3.973527             65500.0
\end{Verbatim}
\end{tcolorbox}
        
    \begin{tcolorbox}[breakable, size=fbox, boxrule=1pt, pad at break*=1mm,colback=cellbackground, colframe=cellborder]
\prompt{In}{incolor}{10}{\boxspacing}
\begin{Verbatim}[commandchars=\\\{\}]
\PY{n}{plt}\PY{o}{.}\PY{n}{figure}\PY{p}{(}\PY{n}{figsize}\PY{o}{=}\PY{p}{(}\PY{l+m+mi}{12}\PY{p}{,}\PY{l+m+mi}{7}\PY{p}{)}\PY{p}{)}
\PY{n}{sns}\PY{o}{.}\PY{n}{scatterplot}\PY{p}{(}\PY{n}{data}\PY{o}{=}\PY{n}{pca\PYZus{}df}\PY{p}{,} \PY{n}{x}\PY{o}{=}\PY{l+s+s1}{\PYZsq{}}\PY{l+s+s1}{PC 1}\PY{l+s+s1}{\PYZsq{}}\PY{p}{,} \PY{n}{y}\PY{o}{=}\PY{l+s+s1}{\PYZsq{}}\PY{l+s+s1}{PC 2}\PY{l+s+s1}{\PYZsq{}}\PY{p}{,} \PY{n}{hue}\PY{o}{=}\PY{l+s+s2}{\PYZdq{}}\PY{l+s+s2}{median\PYZus{}house\PYZus{}value}\PY{l+s+s2}{\PYZdq{}}\PY{p}{)}\PY{p}{;}
\end{Verbatim}
\end{tcolorbox}

    \begin{center}
    \adjustimage{max size={0.9\linewidth}{0.9\paperheight}}{output_20_0.png}
    \end{center}
    { \hspace*{\fill} \\}
    
    Whitening in this example seems to contribute to variance reduction. The
range of our PCs are smaller.

    \begin{enumerate}
\def\labelenumi{(\alph{enumi})}
\setcounter{enumi}{4}
\tightlist
\item
\end{enumerate}

    \begin{tcolorbox}[breakable, size=fbox, boxrule=1pt, pad at break*=1mm,colback=cellbackground, colframe=cellborder]
\prompt{In}{incolor}{11}{\boxspacing}
\begin{Verbatim}[commandchars=\\\{\}]
\PY{n}{pca} \PY{o}{=} \PY{n}{PCA}\PY{p}{(}\PY{n}{n\PYZus{}components}\PY{o}{=}\PY{l+m+mi}{3}\PY{p}{,} \PY{n}{whiten}\PY{o}{=}\PY{k+kc}{True}\PY{p}{)}
\PY{n}{pcs} \PY{o}{=} \PY{n}{pca}\PY{o}{.}\PY{n}{fit\PYZus{}transform}\PY{p}{(}\PY{n}{features}\PY{p}{)}
\PY{n}{pca\PYZus{}df} \PY{o}{=} \PY{n}{pd}\PY{o}{.}\PY{n}{DataFrame}\PY{p}{(}\PY{n}{data}\PY{o}{=}\PY{n}{pcs}\PY{p}{,} \PY{n}{columns} \PY{o}{=} \PY{p}{[}\PY{l+s+s1}{\PYZsq{}}\PY{l+s+s1}{PC 1}\PY{l+s+s1}{\PYZsq{}}\PY{p}{,} \PY{l+s+s1}{\PYZsq{}}\PY{l+s+s1}{PC 2}\PY{l+s+s1}{\PYZsq{}}\PY{p}{,} \PY{l+s+s1}{\PYZsq{}}\PY{l+s+s1}{median\PYZus{}house\PYZus{}value}\PY{l+s+s1}{\PYZsq{}}\PY{p}{]}\PY{p}{)}
\end{Verbatim}
\end{tcolorbox}

    \begin{tcolorbox}[breakable, size=fbox, boxrule=1pt, pad at break*=1mm,colback=cellbackground, colframe=cellborder]
\prompt{In}{incolor}{12}{\boxspacing}
\begin{Verbatim}[commandchars=\\\{\}]
\PY{n}{pca\PYZus{}df}\PY{o}{.}\PY{n}{head}\PY{p}{(}\PY{p}{)}
\end{Verbatim}
\end{tcolorbox}

            \begin{tcolorbox}[breakable, size=fbox, boxrule=.5pt, pad at break*=1mm, opacityfill=0]
\prompt{Out}{outcolor}{12}{\boxspacing}
\begin{Verbatim}[commandchars=\\\{\}]
       PC 1      PC 2  median\_house\_value
0 -1.210522 -1.688243            5.068627
1 -1.096713 -1.590321            5.189372
2 -1.048430 -1.745882            4.094409
3 -1.154480 -1.762187            3.950082
4 -1.222593 -1.783791            3.814070
\end{Verbatim}
\end{tcolorbox}
        
    \begin{tcolorbox}[breakable, size=fbox, boxrule=1pt, pad at break*=1mm,colback=cellbackground, colframe=cellborder]
\prompt{In}{incolor}{13}{\boxspacing}
\begin{Verbatim}[commandchars=\\\{\}]
\PY{n}{plt}\PY{o}{.}\PY{n}{figure}\PY{p}{(}\PY{n}{figsize}\PY{o}{=}\PY{p}{(}\PY{l+m+mi}{12}\PY{p}{,}\PY{l+m+mi}{7}\PY{p}{)}\PY{p}{)}
\PY{n}{sns}\PY{o}{.}\PY{n}{scatterplot}\PY{p}{(}\PY{n}{data}\PY{o}{=}\PY{n}{pca\PYZus{}df}\PY{p}{,} \PY{n}{x}\PY{o}{=}\PY{l+s+s1}{\PYZsq{}}\PY{l+s+s1}{PC 1}\PY{l+s+s1}{\PYZsq{}}\PY{p}{,} \PY{n}{y}\PY{o}{=}\PY{l+s+s1}{\PYZsq{}}\PY{l+s+s1}{PC 2}\PY{l+s+s1}{\PYZsq{}}\PY{p}{,} \PY{n}{hue}\PY{o}{=}\PY{l+s+s2}{\PYZdq{}}\PY{l+s+s2}{median\PYZus{}house\PYZus{}value}\PY{l+s+s2}{\PYZdq{}}\PY{p}{)}\PY{p}{;}
\end{Verbatim}
\end{tcolorbox}

    \begin{center}
    \adjustimage{max size={0.9\linewidth}{0.9\paperheight}}{output_25_0.png}
    \end{center}
    { \hspace*{\fill} \\}
    
    \begin{enumerate}
\def\labelenumi{(\alph{enumi})}
\setcounter{enumi}{5}
\tightlist
\item
  The variation in the median house value in the full dataset PCA case
  above is not simpler than the previous 2 PC case. It isn't simpler
  because there are multiple distinct ranges of PC1 and PC2 combinations
  that equate to the same median house value (e.g.~median house value of
  0 and 2 can be seen in PC1 ranges {[}-1, 2{]} and PC2 ranges {[}-1.5,
  -0.5{]} + {[}0.5, 1.5{]}) Whereas in the previous example, there is a
  gradual transition in median house values over the range of PCs.
\end{enumerate}

    \begin{enumerate}
\def\labelenumi{(\alph{enumi})}
\setcounter{enumi}{6}
\tightlist
\item
  I would use PC2. This is because the median house values shown in the
  graph in the 2 PC example are easily distinguishable when referring to
  the y-component (PC2). We can get a rough but somewhat accurate idea
  of the price group from it alone. Compared to the PC1 axis which has
  one of each price groups (low, med, high) for each level in its entire
  range.
\end{enumerate}

    \hypertarget{clustering}{%
\subsection{Clustering}\label{clustering}}

    \hypertarget{q2}{%
\paragraph{\texorpdfstring{\textbf{Q2}}{Q2}}\label{q2}}

    \textbf{10 points} = \((1 + 2 + 3 + 1.5 + 1.5 + 2)\)

    A bird detection system that is equipped with multiple sensors is
deployed in an observation station in an open space and it detects and
collects some information about birds that visit the space. For each
detected bird, it is able to collect the following information:

\begin{itemize}
\tightlist
\item
  \texttt{position} - a value of the form
  (\texttt{x},\texttt{y},\texttt{z}) which indicates the coordinates in
  a three-dimensional space defined by the LiDAR field of the system.
  \texttt{x} and \texttt{y} coordinates are in the range (-500,500)
  whereas \texttt{z} coordinate is in the range (0,150) where 0
  indicates the ground level for \texttt{z}.
\item
  \texttt{sound\_level} - a value in decibels (0-120 dB) of the sound
  made by the detected bird with 0 indicating no audible sound and 120
  indicating maximum sound that can be detected.
\item
  \texttt{time\_of\_visit} - a value indicating number of milliseconds
  since the 00:00 hours of the day of collection.
\end{itemize}

The system sends out a table with the above three fields periodically to
you. You are leading a team of data scientists to explore if the data
collected by this system can be used to distinguish the different
species of birds visiting the station located in the open space.

\begin{enumerate}
\def\labelenumi{(\alph{enumi})}
\item
  You decide that using clustering for initially attempting this problem
  is a good option. How would you \emph{logically} convince your team of
  this?
\item
  Your team is trying to decide between using K-means and hierarchical
  clustering, both based on an Euclidean distance measure. Propose a
  list of preprocessing operations to be performed on the before you use
  any clustering algorithm on the raw dataset. \textbf{Justify} why you
  include each step.
\item
  An enthusiastic intern in your team selects a subset of \emph{raw}
  data samples from the dataset, selecting some nocturnal birds that
  visit all together exactly at midnight everyday and they cannot
  produce any sound. They all have been observed to consistently sit on
  the ground along a straight line. He generates the following
  dendrogram (left) using an agglomerative hierarchical clustering with
  an appropriate linkage that maximizes intercluster dissimilarity. On
  the right is a top-view visualization of the arrangement of the birds
  in the station, based on the dendrogram. The bird B1 has already been
  placed. Place the birds B2 - B9 on the line with an \emph{appropriate}
  spacing between them. \textbf{Explain} your choice.
\end{enumerate}

\begin{enumerate}
\def\labelenumi{(\alph{enumi})}
\setcounter{enumi}{3}
\item
  You are informed by a group of expert ornitholigists that groups of
  birds that belong to different species, tend to visit the station in
  almost equal numbers per species. Based on this information, what type
  of linkage would you use in the hierarchical clustering algorithm?
  \textbf{Justify} your choice.
\item
  The group of ornithologists have identified that exactly 10 species of
  birds visit the station. Does this information help you decide between
  choosing the K-means and hierarchical clustering algorithms?
  \textbf{Explain} why/why not.
\item
  Briefly outline any \textbf{two} ways in which you can verify if the
  clustering you have performed has captured the natural grouping that
  exists among the actual data samples.
\end{enumerate}

    \begin{enumerate}
\def\labelenumi{(\alph{enumi})}
\tightlist
\item
\end{enumerate}

    Clustering is a good idea here since we are not looking to predict a
specific outcome. Instead we are trying to associate or group new birds
with birds of the same type. We are trying to combine birds of similar
characteristics based on a few recorded features.

    \begin{enumerate}
\def\labelenumi{(\alph{enumi})}
\setcounter{enumi}{1}
\tightlist
\item
\end{enumerate}

    \begin{itemize}
\tightlist
\item
  Remove outliers and impute missing values since KMeans cannot handle
  them.
\item
  Normalize all variables so that euclidean distance isn't influenced in
  large part by a subset of the features.
\end{itemize}

    \begin{enumerate}
\def\labelenumi{(\alph{enumi})}
\setcounter{enumi}{2}
\tightlist
\item
\end{enumerate}
\newpage
\begin{figure}[h]
    \includegraphics[width=8cm]{2c.jpg}
\end{figure}

    \begin{enumerate}
\def\labelenumi{(\alph{enumi})}
\setcounter{enumi}{3}
\tightlist
\item
\end{enumerate}

    Given the above-mentioned facts, I would use the centroid or average
linkage. To assess the similarity or dissimilarity between groups, since
they arrive in groups/clusters, we can take the average of all the birds
of a given category and compare it to another groups average,
essentially implementing distance between centroids as our main metric
to build our hierarchical clustering diagram.

    \begin{enumerate}
\def\labelenumi{(\alph{enumi})}
\setcounter{enumi}{4}
\tightlist
\item
\end{enumerate}

    We are not given very much information and since we have few points and
no guarantee that the data we are dealing with is hyper spherical, it is
somewhat more appropriate to use hierarchical clustering.

    \begin{enumerate}
\def\labelenumi{(\alph{enumi})}
\setcounter{enumi}{5}
\tightlist
\item
\end{enumerate}

    We can use a combination of the following 2 metrics to assess the
accuracy of our clustering.

\begin{itemize}
\tightlist
\item
  Cluster Cohesion: We measure the distance between points within a
  cluster and try to minimize this value.
\item
  Cluster Separation: We measure the distance between clusters and try
  to maximize this value.
\end{itemize}

    \hypertarget{cross-validation}{%
\subsection{Cross Validation}\label{cross-validation}}

    \hypertarget{q3}{%
\paragraph{\texorpdfstring{\textbf{Q3}}{Q3}}\label{q3}}

    \textbf{10 points} = \$(1 + 2 + 3 + 1.5 + 1 + 1.5) \$

    Given below is a pair of plots generated while cross-validating a K-NN
model trained on a dataset with various values of K using 12-fold CV and
Leave-one-out (LOOCV) methods :

Answer the following questions:

\begin{enumerate}
\def\labelenumi{(\alph{enumi})}
\item
  What is the motivation behind using cross-validation techniques such
  as LOOCV and k-fold CV over having a validation set?
\item
  What could possibly explain the difference in the error curves in the
  two methods in the plots?
\item
  Describe how similar (or) different the error curves generated on the
  same dataset would look like for another independent run of
  \textbf{each of} the methods \emph{12-fold} and \emph{LOOCV}, compared
  to the plots above. \textbf{Explain} the reason for your answer in
  each case.
\item
  Under which circumstances would you would favour using LOOCV over
  k-fold CV?
\item
  Based on the above plots, what is the best value for the
  hyperparameter K of the model? \textbf{Explain why.}
\item
  What type of hyperparameter search do the above plots illustrate?
  \textbf{Explain}.
\end{enumerate}

    \begin{enumerate}
\def\labelenumi{(\alph{enumi})}
\tightlist
\item
  The main motivations are that we can reduce the bias in our model and
  that we can utilize the entirety of our training set. Having a
  validation set limits the size of our training set and therefore
  prevents us from extracting all of the insight from our training set.
\end{enumerate}

    \begin{enumerate}
\def\labelenumi{(\alph{enumi})}
\setcounter{enumi}{1}
\tightlist
\item
  The fact that LOOCV has access to a greater proportion of the training
  set per iteration could explain why overall LOOCV seems to be a much
  better performer than 12-fold CV. It will yield approximately unbiased
  estimates compared to the somewhat biased 12-fold CV estimates.
\end{enumerate}

    \begin{enumerate}
\def\labelenumi{(\alph{enumi})}
\setcounter{enumi}{2}
\tightlist
\item
  Assuming we are dealing with independent estimates, the LOOCV curve
  will look similar to the one presented since it will have lower
  variance between models/iterations. We have lower variance since we're
  moving a single data point and so we'll have a very high rate of
  overlap between our training sets. This is not the case however in the
  case of KFold-CV. We can expect to see a quite different curve if we
  run it again.
\end{enumerate}

    \begin{enumerate}
\def\labelenumi{(\alph{enumi})}
\setcounter{enumi}{3}
\tightlist
\item
  If we have a very small dataset, then using LOOCV would be preferable
  over KFold CV since we want to be able to use the most of the training
  set as we can for training.
\end{enumerate}

    \begin{enumerate}
\def\labelenumi{(\alph{enumi})}
\setcounter{enumi}{4}
\tightlist
\item
  A K value of 5 seems appropriate as it yields the lower error in both
  LOOCV and KFold CV. It also happens to be a pretty standard/default
  value for KFold CV.
\end{enumerate}

    \begin{enumerate}
\def\labelenumi{(\alph{enumi})}
\setcounter{enumi}{5}
\tightlist
\item
  These plots demonstrate a manual hyperparameter search. Here, we are
  analysing a single hyperparameter K for both CV methods. We are
  checking to see what error each value yields and combined with our
  understanding of the bias variance tradeoff, are looking to choose the
  optimal K value.
\end{enumerate}

    \hypertarget{inference-and-bootstrapping}{%
\subsection{Inference and
Bootstrapping}\label{inference-and-bootstrapping}}

    \hypertarget{q4}{%
\paragraph{\texorpdfstring{\textbf{Q4}}{Q4}}\label{q4}}

    \textbf{16 points} =
\((1.5 + 1.5 + 1.5 + 1.5 + 2 + 2 + 1.5 + 1.5 + 1.5 + 1.5)\)

    The dataset loaded in the next cell consists of data from a drug trial
experiment.\\
* \texttt{subject\_type} indicates 0 for if a subject is a
\emph{control} and 1 if taking \emph{treatment}. *
\texttt{daily\_dosage} indicates the dosage of the drug in millilitres
(mL) * \texttt{life\_expectancy} show the projected age (year) upto
which that the subject is expected to live.

Let the field \texttt{subject\_type} in the dataset correspond to
\(x_{type}\), \texttt{daily\_dosage} to \(x_{dosage}\) and
\texttt{life\_expectancy} to \(y\).

\begin{enumerate}
\def\labelenumi{(\alph{enumi})}
\tightlist
\item
  Now, consider the regression :
  \[y = \beta_{dosage} x_{dosage} + \beta_{type} x_{type} + \beta_{0} + \epsilon\]
\end{enumerate}

Write your code to perform this regression and list the coefficient
estimates \(\hat{\beta}_{dosage}\), \(\hat{\beta}_{type}\) and
\(\hat{\beta}_{0}\) that you obtained by running your code.

\begin{enumerate}
\def\labelenumi{(\alph{enumi})}
\setcounter{enumi}{1}
\item
  Run a bootstrap of the dataset over 500 iterations, and collect the
  coefficients \(\hat{\beta}_{dosage}\), \(\hat{\beta}_{type}\) and
  \(\hat{\beta}_{0}\) that you obtain in each iteration.
\item
  Use the coefficient estimates that you collected in (b) and estimate
  the standard errors of all the 3 coefficients
  \(S.E.(\hat{\beta}_{dosage})\), \(S.E.(\hat{\beta}_{type})\) and
  \(S.E.(\hat{\beta}_{0})\).
\item
  Plot a histogram to observe the distribution of each of the collected
  coefficient estimates. What do you observe?
\item
  Provide a 95\% confidence interval for each of the coefficient
  estimates. What does this interval mean?
\item
  Generate the scatterplot for the points in the dataset with
  \texttt{daily\_dosage} on the x-axis, \texttt{life\_expectancy} on the
  y-axis and the points colored by the \texttt{subject\_type} value
  (separate colors to indicate the types 0 and 1) with the collected 500
  bootstrap sampled fits overlaid. To make the plot easier to read,
  reduce the transparency of the lines.
\item
  Based on all the above, \textbf{explain} intuitively what is conveyed
  by the plot you generated in (f).
\item
  Make a scatterplot of the bootstrapped coefficients,
  \(\left(\beta_{type}^{\ast}, \beta_{dosage}^{\ast}\right)\) against
  one another. \textbf{Comment} on the overall distribution of these two
  coefficients \textbf{and} the nature of correlation between them.
\item
  Similar to above, estimate the coefficient estimates
  \(\hat{\beta}_{dosage}\) and \(\hat{\beta}_{0}\) and the standard
  error of the dosage coefficient \(S.E.(\hat{\beta}_{dosage})\) by
  bootstrapping the dataset over 500 iterations and fitting the dataset
  in the regression :
  \[y = \beta_{dosage} x_{dosage} + \beta_{0} + \epsilon\]
\item
  Comparing the results in (i) and (c), what can you comment about the
  relationship of \texttt{daily\_dosage} and \texttt{subject\_type} with
  \texttt{life\_expectancy}?
\end{enumerate}

\textbf{Bonus : (3 points)} \textgreater{} (k) In a bootstrap of the
dataset over 500 iterations similar to the above, perform an independent
t-test with an \(\alpha=0.05\) significance level, using
\texttt{scipy.stats.ttest\_ind} on the \texttt{life\_expectancy} of
these two groups . Collect the p-values and plot the p-value histogram.

    \begin{tcolorbox}[breakable, size=fbox, boxrule=1pt, pad at break*=1mm,colback=cellbackground, colframe=cellborder]
\prompt{In}{incolor}{14}{\boxspacing}
\begin{Verbatim}[commandchars=\\\{\}]
\PY{n}{trials} \PY{o}{=} \PY{n}{pd}\PY{o}{.}\PY{n}{read\PYZus{}csv}\PY{p}{(}\PY{n}{path}\PY{o}{.}\PY{n}{format}\PY{p}{(}\PY{l+s+s1}{\PYZsq{}}\PY{l+s+s1}{drug\PYZhy{}trials.csv}\PY{l+s+s1}{\PYZsq{}}\PY{p}{)}\PY{p}{)}
\end{Verbatim}
\end{tcolorbox}

    \begin{enumerate}
\def\labelenumi{(\alph{enumi})}
\tightlist
\item
\end{enumerate}

    \begin{tcolorbox}[breakable, size=fbox, boxrule=1pt, pad at break*=1mm,colback=cellbackground, colframe=cellborder]
\prompt{In}{incolor}{15}{\boxspacing}
\begin{Verbatim}[commandchars=\\\{\}]
\PY{n}{trials}\PY{o}{.}\PY{n}{head}\PY{p}{(}\PY{p}{)}
\end{Verbatim}
\end{tcolorbox}

            \begin{tcolorbox}[breakable, size=fbox, boxrule=.5pt, pad at break*=1mm, opacityfill=0]
\prompt{Out}{outcolor}{15}{\boxspacing}
\begin{Verbatim}[commandchars=\\\{\}]
   subject\_type  daily\_dosage  life\_expectancy
0             0        22.762         63.39666
1             0        22.762         80.14504
2             1        16.036         82.31560
3             1        23.830         84.20141
4             0         3.073         57.28034
\end{Verbatim}
\end{tcolorbox}
        
    \begin{tcolorbox}[breakable, size=fbox, boxrule=1pt, pad at break*=1mm,colback=cellbackground, colframe=cellborder]
\prompt{In}{incolor}{28}{\boxspacing}
\begin{Verbatim}[commandchars=\\\{\}]
\PY{n}{X} \PY{o}{=} \PY{n}{trials}\PY{p}{[}\PY{p}{[}\PY{l+s+s1}{\PYZsq{}}\PY{l+s+s1}{daily\PYZus{}dosage}\PY{l+s+s1}{\PYZsq{}}\PY{p}{,} \PY{l+s+s1}{\PYZsq{}}\PY{l+s+s1}{subject\PYZus{}type}\PY{l+s+s1}{\PYZsq{}}\PY{p}{]}\PY{p}{]}
\PY{n}{y} \PY{o}{=} \PY{n}{trials}\PY{p}{[}\PY{l+s+s1}{\PYZsq{}}\PY{l+s+s1}{life\PYZus{}expectancy}\PY{l+s+s1}{\PYZsq{}}\PY{p}{]}
\PY{n}{lr} \PY{o}{=} \PY{n}{LinearRegression}\PY{p}{(}\PY{p}{)}\PY{o}{.}\PY{n}{fit}\PY{p}{(}\PY{n}{X}\PY{p}{,} \PY{n}{y}\PY{p}{)}
\end{Verbatim}
\end{tcolorbox}

    \begin{tcolorbox}[breakable, size=fbox, boxrule=1pt, pad at break*=1mm,colback=cellbackground, colframe=cellborder]
\prompt{In}{incolor}{29}{\boxspacing}
\begin{Verbatim}[commandchars=\\\{\}]
\PY{n}{lr}\PY{o}{.}\PY{n}{intercept\PYZus{}}
\end{Verbatim}
\end{tcolorbox}

            \begin{tcolorbox}[breakable, size=fbox, boxrule=.5pt, pad at break*=1mm, opacityfill=0]
\prompt{Out}{outcolor}{29}{\boxspacing}
\begin{Verbatim}[commandchars=\\\{\}]
65.67943220081068
\end{Verbatim}
\end{tcolorbox}
        
    \begin{tcolorbox}[breakable, size=fbox, boxrule=1pt, pad at break*=1mm,colback=cellbackground, colframe=cellborder]
\prompt{In}{incolor}{30}{\boxspacing}
\begin{Verbatim}[commandchars=\\\{\}]
\PY{n}{lr}\PY{o}{.}\PY{n}{coef\PYZus{}}
\end{Verbatim}
\end{tcolorbox}

            \begin{tcolorbox}[breakable, size=fbox, boxrule=.5pt, pad at break*=1mm, opacityfill=0]
\prompt{Out}{outcolor}{30}{\boxspacing}
\begin{Verbatim}[commandchars=\\\{\}]
array([0.27154927, 5.6681168 ])
\end{Verbatim}
\end{tcolorbox}
        
    B\_dosage = 0.2715

B\_type = 5.668

B\_0 = 65.6794

    \begin{enumerate}
\def\labelenumi{(\alph{enumi})}
\setcounter{enumi}{1}
\tightlist
\item
\end{enumerate}

    \begin{tcolorbox}[breakable, size=fbox, boxrule=1pt, pad at break*=1mm,colback=cellbackground, colframe=cellborder]
\prompt{In}{incolor}{57}{\boxspacing}
\begin{Verbatim}[commandchars=\\\{\}]
\PY{n}{coef\PYZus{}dosage} \PY{o}{=} \PY{p}{[}\PY{p}{]}
\PY{n}{coef\PYZus{}type} \PY{o}{=} \PY{p}{[}\PY{p}{]}
\PY{n}{coef\PYZus{}0} \PY{o}{=} \PY{p}{[}\PY{p}{]}

\PY{k}{for} \PY{n}{i} \PY{o+ow}{in} \PY{n+nb}{range}\PY{p}{(}\PY{l+m+mi}{500}\PY{p}{)}\PY{p}{:}
    \PY{c+c1}{\PYZsh{} Resample from data with replacement then split into X and y.}
    \PY{n}{trials\PYZus{}boot} \PY{o}{=} \PY{n}{resample}\PY{p}{(}\PY{n}{trials}\PY{p}{,} \PY{n}{replace}\PY{o}{=}\PY{k+kc}{True}\PY{p}{)}
    \PY{n}{X\PYZus{}boot} \PY{o}{=} \PY{n}{trials\PYZus{}boot}\PY{p}{[}\PY{p}{[}\PY{l+s+s1}{\PYZsq{}}\PY{l+s+s1}{daily\PYZus{}dosage}\PY{l+s+s1}{\PYZsq{}}\PY{p}{,} \PY{l+s+s1}{\PYZsq{}}\PY{l+s+s1}{subject\PYZus{}type}\PY{l+s+s1}{\PYZsq{}}\PY{p}{]}\PY{p}{]}
    \PY{n}{y\PYZus{}boot} \PY{o}{=} \PY{n}{trials\PYZus{}boot}\PY{p}{[}\PY{l+s+s1}{\PYZsq{}}\PY{l+s+s1}{life\PYZus{}expectancy}\PY{l+s+s1}{\PYZsq{}}\PY{p}{]}
    
    \PY{c+c1}{\PYZsh{} Fit regressor and store coefs.}
    \PY{n}{lr} \PY{o}{=} \PY{n}{LinearRegression}\PY{p}{(}\PY{p}{)}\PY{o}{.}\PY{n}{fit}\PY{p}{(}\PY{n}{X\PYZus{}boot}\PY{p}{,} \PY{n}{y\PYZus{}boot}\PY{p}{)}
    \PY{n}{coef\PYZus{}dosage}\PY{o}{.}\PY{n}{append}\PY{p}{(}\PY{n}{lr}\PY{o}{.}\PY{n}{coef\PYZus{}}\PY{p}{[}\PY{l+m+mi}{0}\PY{p}{]}\PY{p}{)}
    \PY{n}{coef\PYZus{}type}\PY{o}{.}\PY{n}{append}\PY{p}{(}\PY{n}{lr}\PY{o}{.}\PY{n}{coef\PYZus{}}\PY{p}{[}\PY{l+m+mi}{1}\PY{p}{]}\PY{p}{)}
    \PY{n}{coef\PYZus{}0}\PY{o}{.}\PY{n}{append}\PY{p}{(}\PY{n}{lr}\PY{o}{.}\PY{n}{intercept\PYZus{}}\PY{p}{)}
\end{Verbatim}
\end{tcolorbox}

    \begin{tcolorbox}[breakable, size=fbox, boxrule=1pt, pad at break*=1mm,colback=cellbackground, colframe=cellborder]
\prompt{In}{incolor}{141}{\boxspacing}
\begin{Verbatim}[commandchars=\\\{\}]
\PY{n+nb}{len}\PY{p}{(}\PY{n}{coef\PYZus{}0}\PY{p}{)}\PY{p}{,} \PY{n+nb}{len}\PY{p}{(}\PY{n}{coef\PYZus{}dosage}\PY{p}{)}\PY{p}{,} \PY{n+nb}{len}\PY{p}{(}\PY{n}{coef\PYZus{}type}\PY{p}{)}
\end{Verbatim}
\end{tcolorbox}

            \begin{tcolorbox}[breakable, size=fbox, boxrule=.5pt, pad at break*=1mm, opacityfill=0]
\prompt{Out}{outcolor}{141}{\boxspacing}
\begin{Verbatim}[commandchars=\\\{\}]
(500, 500, 500)
\end{Verbatim}
\end{tcolorbox}
        
    \begin{enumerate}
\def\labelenumi{(\alph{enumi})}
\setcounter{enumi}{2}
\tightlist
\item
\end{enumerate}

    \begin{tcolorbox}[breakable, size=fbox, boxrule=1pt, pad at break*=1mm,colback=cellbackground, colframe=cellborder]
\prompt{In}{incolor}{59}{\boxspacing}
\begin{Verbatim}[commandchars=\\\{\}]
\PY{n}{np}\PY{o}{.}\PY{n}{std}\PY{p}{(}\PY{n}{coef\PYZus{}dosage}\PY{p}{)}
\end{Verbatim}
\end{tcolorbox}

            \begin{tcolorbox}[breakable, size=fbox, boxrule=.5pt, pad at break*=1mm, opacityfill=0]
\prompt{Out}{outcolor}{59}{\boxspacing}
\begin{Verbatim}[commandchars=\\\{\}]
0.11139904514756391
\end{Verbatim}
\end{tcolorbox}
        
    \begin{tcolorbox}[breakable, size=fbox, boxrule=1pt, pad at break*=1mm,colback=cellbackground, colframe=cellborder]
\prompt{In}{incolor}{60}{\boxspacing}
\begin{Verbatim}[commandchars=\\\{\}]
\PY{n}{np}\PY{o}{.}\PY{n}{std}\PY{p}{(}\PY{n}{coef\PYZus{}type}\PY{p}{)}
\end{Verbatim}
\end{tcolorbox}

            \begin{tcolorbox}[breakable, size=fbox, boxrule=.5pt, pad at break*=1mm, opacityfill=0]
\prompt{Out}{outcolor}{60}{\boxspacing}
\begin{Verbatim}[commandchars=\\\{\}]
1.8537206181000243
\end{Verbatim}
\end{tcolorbox}
        
    \begin{tcolorbox}[breakable, size=fbox, boxrule=1pt, pad at break*=1mm,colback=cellbackground, colframe=cellborder]
\prompt{In}{incolor}{61}{\boxspacing}
\begin{Verbatim}[commandchars=\\\{\}]
\PY{n}{np}\PY{o}{.}\PY{n}{std}\PY{p}{(}\PY{n}{coef\PYZus{}0}\PY{p}{)}
\end{Verbatim}
\end{tcolorbox}

            \begin{tcolorbox}[breakable, size=fbox, boxrule=.5pt, pad at break*=1mm, opacityfill=0]
\prompt{Out}{outcolor}{61}{\boxspacing}
\begin{Verbatim}[commandchars=\\\{\}]
2.3145047405328287
\end{Verbatim}
\end{tcolorbox}
        
    \begin{enumerate}
\def\labelenumi{(\alph{enumi})}
\setcounter{enumi}{3}
\tightlist
\item
\end{enumerate}

    \begin{tcolorbox}[breakable, size=fbox, boxrule=1pt, pad at break*=1mm,colback=cellbackground, colframe=cellborder]
\prompt{In}{incolor}{143}{\boxspacing}
\begin{Verbatim}[commandchars=\\\{\}]
\PY{n}{plt}\PY{o}{.}\PY{n}{title}\PY{p}{(}\PY{l+s+s2}{\PYZdq{}}\PY{l+s+s2}{LR Dosage Coefficients from a 500 Iteration Boostrap}\PY{l+s+s2}{\PYZdq{}}\PY{p}{)}
\PY{n}{plt}\PY{o}{.}\PY{n}{hist}\PY{p}{(}\PY{n}{coef\PYZus{}dosage}\PY{p}{)}\PY{p}{;}
\end{Verbatim}
\end{tcolorbox}

    \begin{center}
    \adjustimage{max size={0.9\linewidth}{0.9\paperheight}}{output_73_0.png}
    \end{center}
    { \hspace*{\fill} \\}
    
    \begin{tcolorbox}[breakable, size=fbox, boxrule=1pt, pad at break*=1mm,colback=cellbackground, colframe=cellborder]
\prompt{In}{incolor}{142}{\boxspacing}
\begin{Verbatim}[commandchars=\\\{\}]
\PY{n}{plt}\PY{o}{.}\PY{n}{title}\PY{p}{(}\PY{l+s+s2}{\PYZdq{}}\PY{l+s+s2}{LR Type Coefficients from a 500 Iteration Boostrap}\PY{l+s+s2}{\PYZdq{}}\PY{p}{)}
\PY{n}{plt}\PY{o}{.}\PY{n}{hist}\PY{p}{(}\PY{n}{coef\PYZus{}type}\PY{p}{)}\PY{p}{;}
\end{Verbatim}
\end{tcolorbox}

    \begin{center}
    \adjustimage{max size={0.9\linewidth}{0.9\paperheight}}{output_74_0.png}
    \end{center}
    { \hspace*{\fill} \\}
    
    \begin{tcolorbox}[breakable, size=fbox, boxrule=1pt, pad at break*=1mm,colback=cellbackground, colframe=cellborder]
\prompt{In}{incolor}{70}{\boxspacing}
\begin{Verbatim}[commandchars=\\\{\}]
\PY{n}{plt}\PY{o}{.}\PY{n}{title}\PY{p}{(}\PY{l+s+s2}{\PYZdq{}}\PY{l+s+s2}{Linear Regression Intercepts from a 500 Iteration Boostrap}\PY{l+s+s2}{\PYZdq{}}\PY{p}{)}
\PY{n}{plt}\PY{o}{.}\PY{n}{hist}\PY{p}{(}\PY{n}{coef\PYZus{}0}\PY{p}{)}\PY{p}{;}
\end{Verbatim}
\end{tcolorbox}

    \begin{center}
    \adjustimage{max size={0.9\linewidth}{0.9\paperheight}}{output_75_0.png}
    \end{center}
    { \hspace*{\fill} \\}
    
    Here, I observe that all of the coefficients seem normally distributed.

    \begin{enumerate}
\def\labelenumi{(\alph{enumi})}
\setcounter{enumi}{4}
\tightlist
\item
\end{enumerate}

    \begin{tcolorbox}[breakable, size=fbox, boxrule=1pt, pad at break*=1mm,colback=cellbackground, colframe=cellborder]
\prompt{In}{incolor}{71}{\boxspacing}
\begin{Verbatim}[commandchars=\\\{\}]
\PY{p}{[}\PY{n}{np}\PY{o}{.}\PY{n}{mean}\PY{p}{(}\PY{n}{coef\PYZus{}dosage}\PY{p}{)} \PY{o}{\PYZhy{}} \PY{l+m+mf}{1.96}\PY{o}{*}\PY{n}{np}\PY{o}{.}\PY{n}{std}\PY{p}{(}\PY{n}{coef\PYZus{}dosage}\PY{p}{)}\PY{p}{,} \PY{n}{np}\PY{o}{.}\PY{n}{mean}\PY{p}{(}\PY{n}{coef\PYZus{}dosage}\PY{p}{)} \PY{o}{+} \PY{l+m+mf}{1.96}\PY{o}{*}\PY{n}{np}\PY{o}{.}\PY{n}{std}\PY{p}{(}\PY{n}{coef\PYZus{}dosage}\PY{p}{)}\PY{p}{]}
\end{Verbatim}
\end{tcolorbox}

            \begin{tcolorbox}[breakable, size=fbox, boxrule=.5pt, pad at break*=1mm, opacityfill=0]
\prompt{Out}{outcolor}{71}{\boxspacing}
\begin{Verbatim}[commandchars=\\\{\}]
[0.04843414356615369, 0.4851184005446042]
\end{Verbatim}
\end{tcolorbox}
        
    \begin{tcolorbox}[breakable, size=fbox, boxrule=1pt, pad at break*=1mm,colback=cellbackground, colframe=cellborder]
\prompt{In}{incolor}{72}{\boxspacing}
\begin{Verbatim}[commandchars=\\\{\}]
\PY{p}{[}\PY{n}{np}\PY{o}{.}\PY{n}{mean}\PY{p}{(}\PY{n}{coef\PYZus{}type}\PY{p}{)} \PY{o}{\PYZhy{}} \PY{l+m+mf}{1.96}\PY{o}{*}\PY{n}{np}\PY{o}{.}\PY{n}{std}\PY{p}{(}\PY{n}{coef\PYZus{}type}\PY{p}{)}\PY{p}{,} \PY{n}{np}\PY{o}{.}\PY{n}{mean}\PY{p}{(}\PY{n}{coef\PYZus{}type}\PY{p}{)} \PY{o}{+} \PY{l+m+mf}{1.96}\PY{o}{*}\PY{n}{np}\PY{o}{.}\PY{n}{std}\PY{p}{(}\PY{n}{coef\PYZus{}type}\PY{p}{)}\PY{p}{]}
\end{Verbatim}
\end{tcolorbox}

            \begin{tcolorbox}[breakable, size=fbox, boxrule=.5pt, pad at break*=1mm, opacityfill=0]
\prompt{Out}{outcolor}{72}{\boxspacing}
\begin{Verbatim}[commandchars=\\\{\}]
[2.023685554532826, 9.290270377484921]
\end{Verbatim}
\end{tcolorbox}
        
    \begin{tcolorbox}[breakable, size=fbox, boxrule=1pt, pad at break*=1mm,colback=cellbackground, colframe=cellborder]
\prompt{In}{incolor}{73}{\boxspacing}
\begin{Verbatim}[commandchars=\\\{\}]
\PY{p}{[}\PY{n}{np}\PY{o}{.}\PY{n}{mean}\PY{p}{(}\PY{n}{coef\PYZus{}0}\PY{p}{)} \PY{o}{\PYZhy{}} \PY{l+m+mf}{1.96}\PY{o}{*}\PY{n}{np}\PY{o}{.}\PY{n}{std}\PY{p}{(}\PY{n}{coef\PYZus{}0}\PY{p}{)}\PY{p}{,} \PY{n}{np}\PY{o}{.}\PY{n}{mean}\PY{p}{(}\PY{n}{coef\PYZus{}0}\PY{p}{)} \PY{o}{+} \PY{l+m+mf}{1.96}\PY{o}{*}\PY{n}{np}\PY{o}{.}\PY{n}{std}\PY{p}{(}\PY{n}{coef\PYZus{}0}\PY{p}{)}\PY{p}{]}
\end{Verbatim}
\end{tcolorbox}

            \begin{tcolorbox}[breakable, size=fbox, boxrule=.5pt, pad at break*=1mm, opacityfill=0]
\prompt{Out}{outcolor}{73}{\boxspacing}
\begin{Verbatim}[commandchars=\\\{\}]
[61.24290230798341, 70.3157608908721]
\end{Verbatim}
\end{tcolorbox}
        
    These intervals represent the range in which the values of each
coefficient will fall into, in 95\% of cases. It gives us a reasonable
expectation of the coef values we should be getting.

    \begin{enumerate}
\def\labelenumi{(\alph{enumi})}
\setcounter{enumi}{5}
\tightlist
\item
\end{enumerate}

    \begin{tcolorbox}[breakable, size=fbox, boxrule=1pt, pad at break*=1mm,colback=cellbackground, colframe=cellborder]
\prompt{In}{incolor}{93}{\boxspacing}
\begin{Verbatim}[commandchars=\\\{\}]
\PY{n}{plt}\PY{o}{.}\PY{n}{figure}\PY{p}{(}\PY{n}{figsize}\PY{o}{=}\PY{p}{(}\PY{l+m+mi}{10}\PY{p}{,}\PY{l+m+mi}{7}\PY{p}{)}\PY{p}{)}
\PY{k}{for} \PY{n}{i} \PY{o+ow}{in} \PY{n+nb}{range}\PY{p}{(}\PY{l+m+mi}{500}\PY{p}{)}\PY{p}{:}
    \PY{n}{l} \PY{o}{=} \PY{n}{sns}\PY{o}{.}\PY{n}{lineplot}\PY{p}{(}\PY{n}{x}\PY{o}{=}\PY{n}{trials}\PY{p}{[}\PY{l+s+s1}{\PYZsq{}}\PY{l+s+s1}{daily\PYZus{}dosage}\PY{l+s+s1}{\PYZsq{}}\PY{p}{]}\PY{p}{,} 
                 \PY{n}{y}\PY{o}{=}\PY{n}{coef\PYZus{}dosage}\PY{p}{[}\PY{n}{i}\PY{p}{]}\PY{o}{*}\PY{n}{trials}\PY{p}{[}\PY{l+s+s1}{\PYZsq{}}\PY{l+s+s1}{daily\PYZus{}dosage}\PY{l+s+s1}{\PYZsq{}}\PY{p}{]} \PY{o}{+} \PY{n}{coef\PYZus{}type}\PY{p}{[}\PY{n}{i}\PY{p}{]}\PY{o}{*}\PY{n}{trials}\PY{p}{[}\PY{l+s+s1}{\PYZsq{}}\PY{l+s+s1}{subject\PYZus{}type}\PY{l+s+s1}{\PYZsq{}}\PY{p}{]} \PY{o}{+} \PY{n}{coef\PYZus{}0}\PY{p}{[}\PY{n}{i}\PY{p}{]}\PY{p}{,}
                 \PY{n}{hue}\PY{o}{=}\PY{n}{trials}\PY{p}{[}\PY{l+s+s1}{\PYZsq{}}\PY{l+s+s1}{subject\PYZus{}type}\PY{l+s+s1}{\PYZsq{}}\PY{p}{]}\PY{p}{,}
                 \PY{n}{alpha}\PY{o}{=}\PY{l+m+mf}{0.2}\PY{p}{,}
                 \PY{n}{legend}\PY{o}{=}\PY{k+kc}{False}\PY{p}{)}
\PY{n}{sns}\PY{o}{.}\PY{n}{scatterplot}\PY{p}{(}\PY{n}{x}\PY{o}{=}\PY{n}{trials}\PY{p}{[}\PY{l+s+s1}{\PYZsq{}}\PY{l+s+s1}{daily\PYZus{}dosage}\PY{l+s+s1}{\PYZsq{}}\PY{p}{]}\PY{p}{,} \PY{n}{y}\PY{o}{=}\PY{n}{trials}\PY{p}{[}\PY{l+s+s1}{\PYZsq{}}\PY{l+s+s1}{life\PYZus{}expectancy}\PY{l+s+s1}{\PYZsq{}}\PY{p}{]}\PY{p}{,} \PY{n}{hue}\PY{o}{=}\PY{n}{trials}\PY{p}{[}\PY{l+s+s1}{\PYZsq{}}\PY{l+s+s1}{subject\PYZus{}type}\PY{l+s+s1}{\PYZsq{}}\PY{p}{]}\PY{p}{)}\PY{p}{;}
\end{Verbatim}
\end{tcolorbox}

    \begin{center}
    \adjustimage{max size={0.9\linewidth}{0.9\paperheight}}{output_83_0.png}
    \end{center}
    { \hspace*{\fill} \\}
    
    \begin{enumerate}
\def\labelenumi{(\alph{enumi})}
\setcounter{enumi}{6}
\tightlist
\item
\end{enumerate}

    What we can infer intuitively from this graph is that, generally
speaking, subject\_type = 1 have a higher life expectancy than blue and
coupled with a higher daily dosage, we maximize life expectancy, so
daily dosage is positively correlated to life expectancy.

    \begin{enumerate}
\def\labelenumi{(\alph{enumi})}
\setcounter{enumi}{7}
\tightlist
\item
\end{enumerate}

    \begin{tcolorbox}[breakable, size=fbox, boxrule=1pt, pad at break*=1mm,colback=cellbackground, colframe=cellborder]
\prompt{In}{incolor}{97}{\boxspacing}
\begin{Verbatim}[commandchars=\\\{\}]
\PY{n}{sns}\PY{o}{.}\PY{n}{scatterplot}\PY{p}{(}\PY{n}{x}\PY{o}{=}\PY{n}{coef\PYZus{}type}\PY{p}{,} \PY{n}{y}\PY{o}{=}\PY{n}{coef\PYZus{}dosage}\PY{p}{)}\PY{p}{;}
\end{Verbatim}
\end{tcolorbox}

    \begin{center}
    \adjustimage{max size={0.9\linewidth}{0.9\paperheight}}{output_87_0.png}
    \end{center}
    { \hspace*{\fill} \\}
    
    From this plot of coefficients we can clearly observe that our daily
dosage and subject type coefficients are not correlated. Therefore, this
tells us that our features are not correlated, which removes the worry
of multicollinearity in our model.

    \begin{enumerate}
\def\labelenumi{(\roman{enumi})}
\tightlist
\item
\end{enumerate}

    \begin{tcolorbox}[breakable, size=fbox, boxrule=1pt, pad at break*=1mm,colback=cellbackground, colframe=cellborder]
\prompt{In}{incolor}{115}{\boxspacing}
\begin{Verbatim}[commandchars=\\\{\}]
\PY{n}{coef\PYZus{}dosage} \PY{o}{=} \PY{p}{[}\PY{p}{]}
\PY{n}{coef\PYZus{}0} \PY{o}{=} \PY{p}{[}\PY{p}{]}

\PY{k}{for} \PY{n}{i} \PY{o+ow}{in} \PY{n+nb}{range}\PY{p}{(}\PY{l+m+mi}{500}\PY{p}{)}\PY{p}{:}
    \PY{c+c1}{\PYZsh{} Resample from data with replacement then split into X and y.}
    \PY{n}{trials\PYZus{}boot} \PY{o}{=} \PY{n}{resample}\PY{p}{(}\PY{n}{trials}\PY{p}{,} \PY{n}{replace}\PY{o}{=}\PY{k+kc}{True}\PY{p}{)}
    \PY{n}{X\PYZus{}boot} \PY{o}{=} \PY{n}{np}\PY{o}{.}\PY{n}{array}\PY{p}{(}\PY{n}{trials\PYZus{}boot}\PY{p}{[}\PY{p}{[}\PY{l+s+s1}{\PYZsq{}}\PY{l+s+s1}{daily\PYZus{}dosage}\PY{l+s+s1}{\PYZsq{}}\PY{p}{]}\PY{p}{]}\PY{p}{)}\PY{o}{.}\PY{n}{reshape}\PY{p}{(}\PY{p}{(}\PY{o}{\PYZhy{}}\PY{l+m+mi}{1}\PY{p}{,} \PY{l+m+mi}{1}\PY{p}{)}\PY{p}{)}
    \PY{n}{y\PYZus{}boot} \PY{o}{=} \PY{n}{trials\PYZus{}boot}\PY{p}{[}\PY{p}{[}\PY{l+s+s1}{\PYZsq{}}\PY{l+s+s1}{life\PYZus{}expectancy}\PY{l+s+s1}{\PYZsq{}}\PY{p}{]}\PY{p}{]}
    
    \PY{c+c1}{\PYZsh{} Fit regressor and store coefs.}
    \PY{n}{lr} \PY{o}{=} \PY{n}{LinearRegression}\PY{p}{(}\PY{p}{)}\PY{o}{.}\PY{n}{fit}\PY{p}{(}\PY{n}{X\PYZus{}boot}\PY{p}{,} \PY{n}{y\PYZus{}boot}\PY{p}{)}
    \PY{n}{coef\PYZus{}dosage}\PY{o}{.}\PY{n}{append}\PY{p}{(}\PY{n}{lr}\PY{o}{.}\PY{n}{coef\PYZus{}}\PY{p}{[}\PY{l+m+mi}{0}\PY{p}{]}\PY{p}{[}\PY{l+m+mi}{0}\PY{p}{]}\PY{p}{)}
    \PY{n}{coef\PYZus{}0}\PY{o}{.}\PY{n}{append}\PY{p}{(}\PY{n}{lr}\PY{o}{.}\PY{n}{intercept\PYZus{}}\PY{p}{[}\PY{l+m+mi}{0}\PY{p}{]}\PY{p}{)}
\end{Verbatim}
\end{tcolorbox}

    \begin{tcolorbox}[breakable, size=fbox, boxrule=1pt, pad at break*=1mm,colback=cellbackground, colframe=cellborder]
\prompt{In}{incolor}{146}{\boxspacing}
\begin{Verbatim}[commandchars=\\\{\}]
\PY{n}{np}\PY{o}{.}\PY{n}{mean}\PY{p}{(}\PY{n}{coef\PYZus{}type}\PY{p}{)}
\end{Verbatim}
\end{tcolorbox}

            \begin{tcolorbox}[breakable, size=fbox, boxrule=.5pt, pad at break*=1mm, opacityfill=0]
\prompt{Out}{outcolor}{146}{\boxspacing}
\begin{Verbatim}[commandchars=\\\{\}]
5.656977966008873
\end{Verbatim}
\end{tcolorbox}
        
    \begin{tcolorbox}[breakable, size=fbox, boxrule=1pt, pad at break*=1mm,colback=cellbackground, colframe=cellborder]
\prompt{In}{incolor}{118}{\boxspacing}
\begin{Verbatim}[commandchars=\\\{\}]
\PY{n}{np}\PY{o}{.}\PY{n}{std}\PY{p}{(}\PY{n}{coef\PYZus{}dosage}\PY{p}{)}
\end{Verbatim}
\end{tcolorbox}

            \begin{tcolorbox}[breakable, size=fbox, boxrule=.5pt, pad at break*=1mm, opacityfill=0]
\prompt{Out}{outcolor}{118}{\boxspacing}
\begin{Verbatim}[commandchars=\\\{\}]
0.11428382812749135
\end{Verbatim}
\end{tcolorbox}
        
    \begin{enumerate}
\def\labelenumi{(\alph{enumi})}
\setcounter{enumi}{9}
\tightlist
\item
  We can confirm that both variables are positively correlated with life
  expectancy. For every unit of daily dosage, we have an average of 0.3
  unit increase in life expectancy and for subject type 1 we have an
  average of 5.6 unit increase in life expectancy.
\end{enumerate}

    \hypertarget{feature-engineering}{%
\subsection{Feature Engineering}\label{feature-engineering}}

    \hypertarget{q5}{%
\paragraph{\texorpdfstring{\textbf{Q5}}{Q5}}\label{q5}}

    \textbf{4 points} = \((1.5 + 2.5)\)

    Given below is a cell that loads a dataset that contains features
representing the body measurements of certain types of sharks in various
regions in Canada.

Using programming, perform analyses using the following methods to
identify the outlier samples in the dataset :

\begin{enumerate}
\def\labelenumi{(\alph{enumi})}
\item
  For each \emph{feature} among \texttt{body\_length},
  \texttt{fin\_length} and \texttt{tail\_length} in the dataset, use a
  box plot to visualize the feature values (along y-axis) grouped by
  \texttt{region} feature (show on x-axis). Identify the \texttt{region}
  and \texttt{feature} (other than \texttt{region}) that shows the
  highest number of outliers.
\item
  For the \texttt{region} and the \texttt{feature} you chose in (a), use
  the Q3-Q1 Inter-Quartile Range (IQR) to identify and list the rows of
  the outliers from the dataframe.
\end{enumerate}

    \begin{tcolorbox}[breakable, size=fbox, boxrule=1pt, pad at break*=1mm,colback=cellbackground, colframe=cellborder]
\prompt{In}{incolor}{149}{\boxspacing}
\begin{Verbatim}[commandchars=\\\{\}]
\PY{n}{sharks} \PY{o}{=} \PY{n}{pd}\PY{o}{.}\PY{n}{read\PYZus{}csv}\PY{p}{(}\PY{n}{path}\PY{o}{.}\PY{n}{format}\PY{p}{(}\PY{l+s+s1}{\PYZsq{}}\PY{l+s+s1}{ca\PYZhy{}sharks.csv}\PY{l+s+s1}{\PYZsq{}}\PY{p}{)}\PY{p}{)}
\end{Verbatim}
\end{tcolorbox}

    \begin{enumerate}
\def\labelenumi{(\alph{enumi})}
\tightlist
\item
\end{enumerate}

    \begin{tcolorbox}[breakable, size=fbox, boxrule=1pt, pad at break*=1mm,colback=cellbackground, colframe=cellborder]
\prompt{In}{incolor}{150}{\boxspacing}
\begin{Verbatim}[commandchars=\\\{\}]
\PY{k}{for} \PY{n}{f} \PY{o+ow}{in} \PY{p}{[}\PY{l+s+s1}{\PYZsq{}}\PY{l+s+s1}{body\PYZus{}length}\PY{l+s+s1}{\PYZsq{}}\PY{p}{,} \PY{l+s+s1}{\PYZsq{}}\PY{l+s+s1}{fin\PYZus{}length}\PY{l+s+s1}{\PYZsq{}}\PY{p}{,} \PY{l+s+s1}{\PYZsq{}}\PY{l+s+s1}{tail\PYZus{}length}\PY{l+s+s1}{\PYZsq{}}\PY{p}{]}\PY{p}{:}
    \PY{n}{sns}\PY{o}{.}\PY{n}{boxplot}\PY{p}{(}\PY{n}{x}\PY{o}{=}\PY{l+s+s1}{\PYZsq{}}\PY{l+s+s1}{region}\PY{l+s+s1}{\PYZsq{}}\PY{p}{,} \PY{n}{y}\PY{o}{=}\PY{n}{f}\PY{p}{,} \PY{n}{data}\PY{o}{=}\PY{n}{sharks}\PY{p}{)}
    \PY{n}{plt}\PY{o}{.}\PY{n}{show}\PY{p}{(}\PY{p}{)}\PY{p}{;}
\end{Verbatim}
\end{tcolorbox}

    \begin{center}
    \adjustimage{max size={0.9\linewidth}{0.9\paperheight}}{output_100_0.png}
    \end{center}
    { \hspace*{\fill} \\}
    
    \begin{center}
    \adjustimage{max size={0.9\linewidth}{0.9\paperheight}}{output_100_1.png}
    \end{center}
    { \hspace*{\fill} \\}
    
    \begin{center}
    \adjustimage{max size={0.9\linewidth}{0.9\paperheight}}{output_100_2.png}
    \end{center}
    { \hspace*{\fill} \\}
    
    The body length feature shows the most number of outliers, namely in the
YT region. (There are a small amount of outliers in AB region as well)

    \begin{enumerate}
\def\labelenumi{(\alph{enumi})}
\setcounter{enumi}{1}
\tightlist
\item
\end{enumerate}

    \begin{tcolorbox}[breakable, size=fbox, boxrule=1pt, pad at break*=1mm,colback=cellbackground, colframe=cellborder]
\prompt{In}{incolor}{161}{\boxspacing}
\begin{Verbatim}[commandchars=\\\{\}]
\PY{n}{Q1} \PY{o}{=} \PY{n}{np}\PY{o}{.}\PY{n}{percentile}\PY{p}{(}\PY{n}{sharks}\PY{p}{[}\PY{n}{sharks}\PY{p}{[}\PY{l+s+s1}{\PYZsq{}}\PY{l+s+s1}{region}\PY{l+s+s1}{\PYZsq{}}\PY{p}{]}\PY{o}{==}\PY{l+s+s1}{\PYZsq{}}\PY{l+s+s1}{YT}\PY{l+s+s1}{\PYZsq{}}\PY{p}{]}\PY{p}{[}\PY{l+s+s1}{\PYZsq{}}\PY{l+s+s1}{body\PYZus{}length}\PY{l+s+s1}{\PYZsq{}}\PY{p}{]}\PY{p}{,} \PY{l+m+mi}{25}\PY{p}{)}
\PY{n}{Q3} \PY{o}{=} \PY{n}{np}\PY{o}{.}\PY{n}{percentile}\PY{p}{(}\PY{n}{sharks}\PY{p}{[}\PY{n}{sharks}\PY{p}{[}\PY{l+s+s1}{\PYZsq{}}\PY{l+s+s1}{region}\PY{l+s+s1}{\PYZsq{}}\PY{p}{]}\PY{o}{==}\PY{l+s+s1}{\PYZsq{}}\PY{l+s+s1}{YT}\PY{l+s+s1}{\PYZsq{}}\PY{p}{]}\PY{p}{[}\PY{l+s+s1}{\PYZsq{}}\PY{l+s+s1}{body\PYZus{}length}\PY{l+s+s1}{\PYZsq{}}\PY{p}{]}\PY{p}{,} \PY{l+m+mi}{75}\PY{p}{)}
\PY{n}{IQR} \PY{o}{=} \PY{n}{Q3}\PY{o}{\PYZhy{}}\PY{n}{Q1}
\end{Verbatim}
\end{tcolorbox}

    \begin{tcolorbox}[breakable, size=fbox, boxrule=1pt, pad at break*=1mm,colback=cellbackground, colframe=cellborder]
\prompt{In}{incolor}{162}{\boxspacing}
\begin{Verbatim}[commandchars=\\\{\}]
\PY{n}{IQR}
\end{Verbatim}
\end{tcolorbox}

            \begin{tcolorbox}[breakable, size=fbox, boxrule=.5pt, pad at break*=1mm, opacityfill=0]
\prompt{Out}{outcolor}{162}{\boxspacing}
\begin{Verbatim}[commandchars=\\\{\}]
63.73342500000001
\end{Verbatim}
\end{tcolorbox}
        
    \begin{tcolorbox}[breakable, size=fbox, boxrule=1pt, pad at break*=1mm,colback=cellbackground, colframe=cellborder]
\prompt{In}{incolor}{163}{\boxspacing}
\begin{Verbatim}[commandchars=\\\{\}]
\PY{n}{outliers} \PY{o}{=} \PY{p}{[}\PY{n}{bl} \PY{k}{for} \PY{n}{bl} \PY{o+ow}{in} \PY{n}{sharks}\PY{p}{[}\PY{n}{sharks}\PY{p}{[}\PY{l+s+s1}{\PYZsq{}}\PY{l+s+s1}{region}\PY{l+s+s1}{\PYZsq{}}\PY{p}{]}\PY{o}{==}\PY{l+s+s1}{\PYZsq{}}\PY{l+s+s1}{YT}\PY{l+s+s1}{\PYZsq{}}\PY{p}{]}\PY{p}{[}\PY{l+s+s1}{\PYZsq{}}\PY{l+s+s1}{body\PYZus{}length}\PY{l+s+s1}{\PYZsq{}}\PY{p}{]} \PY{k}{if} \PY{n}{bl} \PY{o}{\PYZlt{}}\PY{o}{=} \PY{n}{Q1} \PY{o}{\PYZhy{}} \PY{l+m+mf}{1.5}\PY{o}{*}\PY{n}{IQR} \PY{o+ow}{or} \PY{n}{bl} \PY{o}{\PYZgt{}}\PY{o}{=} \PY{n}{Q3} \PY{o}{+} \PY{l+m+mf}{1.5}\PY{o}{*}\PY{n}{IQR}\PY{p}{]}
\end{Verbatim}
\end{tcolorbox}

    \begin{tcolorbox}[breakable, size=fbox, boxrule=1pt, pad at break*=1mm,colback=cellbackground, colframe=cellborder]
\prompt{In}{incolor}{164}{\boxspacing}
\begin{Verbatim}[commandchars=\\\{\}]
\PY{n}{outlier\PYZus{}subset} \PY{o}{=} \PY{n}{sharks}\PY{p}{[}\PY{p}{(}\PY{n}{sharks}\PY{p}{[}\PY{l+s+s1}{\PYZsq{}}\PY{l+s+s1}{region}\PY{l+s+s1}{\PYZsq{}}\PY{p}{]}\PY{o}{==}\PY{l+s+s1}{\PYZsq{}}\PY{l+s+s1}{YT}\PY{l+s+s1}{\PYZsq{}}\PY{p}{)} \PY{o}{\PYZam{}} \PY{p}{(}\PY{n}{sharks}\PY{p}{[}\PY{l+s+s1}{\PYZsq{}}\PY{l+s+s1}{body\PYZus{}length}\PY{l+s+s1}{\PYZsq{}}\PY{p}{]}\PY{o}{.}\PY{n}{isin}\PY{p}{(}\PY{n}{outliers}\PY{p}{)}\PY{p}{)}\PY{p}{]}
\end{Verbatim}
\end{tcolorbox}

    \begin{tcolorbox}[breakable, size=fbox, boxrule=1pt, pad at break*=1mm,colback=cellbackground, colframe=cellborder]
\prompt{In}{incolor}{165}{\boxspacing}
\begin{Verbatim}[commandchars=\\\{\}]
\PY{n}{outlier\PYZus{}subset}
\end{Verbatim}
\end{tcolorbox}

            \begin{tcolorbox}[breakable, size=fbox, boxrule=.5pt, pad at break*=1mm, opacityfill=0]
\prompt{Out}{outcolor}{165}{\boxspacing}
\begin{Verbatim}[commandchars=\\\{\}]
    id region  fin\_length  body\_length  tail\_length
35  36     YT        20.2     272.3920     2.653751
53  54     YT        57.8     570.5584     3.253912
54  55     YT        74.0     639.7450     3.365487
\end{Verbatim}
\end{tcolorbox}
        
    \hypertarget{q6}{%
\paragraph{\texorpdfstring{\textbf{Q6}}{Q6}}\label{q6}}

    \textbf{8 points} = \((1 + 3 + 2 + 2 )\)

    For the questions below, answer briefly by inspecting the dataset below.
(There is no need to use any programming) :

This is a representative subset of a collected dataset with information
about used buses across three Canadian cities. A model needs to be fit
to predict the selling price \texttt{Price(\$)} of a bus. \texttt{-\/-}
indicates that the information is not available.

% \begin{longtable}[]{@{}llllllllll@{}}
% \toprule
% \begin{minipage}[b]{(\columnwidth - 9\tabcolsep) * \real{0.10}}\raggedright
% ID\strut
% \end{minipage} &
% \begin{minipage}[b]{(\columnwidth - 9\tabcolsep) * \real{0.10}}\raggedright
% City\strut
% \end{minipage} &
% \begin{minipage}[b]{(\columnwidth - 9\tabcolsep) * \real{0.10}}\raggedright
% Province\strut
% \end{minipage} &
% \begin{minipage}[b]{(\columnwidth - 9\tabcolsep) * \real{0.10}}\raggedright
% Vehicle model\strut
% \end{minipage} &
% \begin{minipage}[b]{(\columnwidth - 9\tabcolsep) * \real{0.10}}\raggedright
% Category\strut
% \end{minipage} &
% \begin{minipage}[b]{(\columnwidth - 9\tabcolsep) * \real{0.10}}\raggedright
% Mileage(kmpl)\strut
% \end{minipage} &
% \begin{minipage}[b]{(\columnwidth - 9\tabcolsep) * \real{0.10}}\raggedright
% Certification\strut
% \end{minipage} &
% \begin{minipage}[b]{(\columnwidth - 9\tabcolsep) * \real{0.10}}\raggedright
% Year\strut
% \end{minipage} &
% \begin{minipage}[b]{(\columnwidth - 9\tabcolsep) * \real{0.10}}\raggedright
% Num\_Damages\strut
% \end{minipage} &
% \begin{minipage}[b]{(\columnwidth - 9\tabcolsep) * \real{0.10}}\raggedright
% Price(\$)\strut
% \end{minipage}\tabularnewline
% \midrule
% \endhead
% \begin{minipage}[t]{(\columnwidth - 9\tabcolsep) * \real{0.10}}\raggedright
% 1\strut
% \end{minipage} &
% \begin{minipage}[t]{(\columnwidth - 9\tabcolsep) * \real{0.10}}\raggedright
% Montréal\strut
% \end{minipage} &
% \begin{minipage}[t]{(\columnwidth - 9\tabcolsep) * \real{0.10}}\raggedright
% QC\strut
% \end{minipage} &
% \begin{minipage}[t]{(\columnwidth - 9\tabcolsep) * \real{0.10}}\raggedright
% CX-120\strut
% \end{minipage} &
% \begin{minipage}[t]{(\columnwidth - 9\tabcolsep) * \real{0.10}}\raggedright
% School\strut
% \end{minipage} &
% \begin{minipage}[t]{(\columnwidth - 9\tabcolsep) * \real{0.10}}\raggedright
% 2.5\strut
% \end{minipage} &
% \begin{minipage}[t]{(\columnwidth - 9\tabcolsep) * \real{0.10}}\raggedright
% Certified\strut
% \end{minipage} &
% \begin{minipage}[t]{(\columnwidth - 9\tabcolsep) * \real{0.10}}\raggedright
% 2018\strut
% \end{minipage} &
% \begin{minipage}[t]{(\columnwidth - 9\tabcolsep) * \real{0.10}}\raggedright
% 3\strut
% \end{minipage} &
% \begin{minipage}[t]{(\columnwidth - 9\tabcolsep) * \real{0.10}}\raggedright
% 44900\strut
% \end{minipage}\tabularnewline
% \begin{minipage}[t]{(\columnwidth - 9\tabcolsep) * \real{0.10}}\raggedright
% 2\strut
% \end{minipage} &
% \begin{minipage}[t]{(\columnwidth - 9\tabcolsep) * \real{0.10}}\raggedright
% Vancouver\strut
% \end{minipage} &
% \begin{minipage}[t]{(\columnwidth - 9\tabcolsep) * \real{0.10}}\raggedright
% BC\strut
% \end{minipage} &
% \begin{minipage}[t]{(\columnwidth - 9\tabcolsep) * \real{0.10}}\raggedright
% AL-100\strut
% \end{minipage} &
% \begin{minipage}[t]{(\columnwidth - 9\tabcolsep) * \real{0.10}}\raggedright
% Tourism\strut
% \end{minipage} &
% \begin{minipage}[t]{(\columnwidth - 9\tabcolsep) * \real{0.10}}\raggedright
% 5\strut
% \end{minipage} &
% \begin{minipage}[t]{(\columnwidth - 9\tabcolsep) * \real{0.10}}\raggedright
% Not certified\strut
% \end{minipage} &
% \begin{minipage}[t]{(\columnwidth - 9\tabcolsep) * \real{0.10}}\raggedright
% 2016\strut
% \end{minipage} &
% \begin{minipage}[t]{(\columnwidth - 9\tabcolsep) * \real{0.10}}\raggedright
% --\strut
% \end{minipage} &
% \begin{minipage}[t]{(\columnwidth - 9\tabcolsep) * \real{0.10}}\raggedright
% 22380\strut
% \end{minipage}\tabularnewline
% \begin{minipage}[t]{(\columnwidth - 9\tabcolsep) * \real{0.10}}\raggedright
% 3\strut
% \end{minipage} &
% \begin{minipage}[t]{(\columnwidth - 9\tabcolsep) * \real{0.10}}\raggedright
% Toronto\strut
% \end{minipage} &
% \begin{minipage}[t]{(\columnwidth - 9\tabcolsep) * \real{0.10}}\raggedright
% ON\strut
% \end{minipage} &
% \begin{minipage}[t]{(\columnwidth - 9\tabcolsep) * \real{0.10}}\raggedright
% WS-978\strut
% \end{minipage} &
% \begin{minipage}[t]{(\columnwidth - 9\tabcolsep) * \real{0.10}}\raggedright
% Tourism\strut
% \end{minipage} &
% \begin{minipage}[t]{(\columnwidth - 9\tabcolsep) * \real{0.10}}\raggedright
% --\strut
% \end{minipage} &
% \begin{minipage}[t]{(\columnwidth - 9\tabcolsep) * \real{0.10}}\raggedright
% Certified\strut
% \end{minipage} &
% \begin{minipage}[t]{(\columnwidth - 9\tabcolsep) * \real{0.10}}\raggedright
% --\strut
% \end{minipage} &
% \begin{minipage}[t]{(\columnwidth - 9\tabcolsep) * \real{0.10}}\raggedright
% --\strut
% \end{minipage} &
% \begin{minipage}[t]{(\columnwidth - 9\tabcolsep) * \real{0.10}}\raggedright
% 30000\strut
% \end{minipage}\tabularnewline
% \begin{minipage}[t]{(\columnwidth - 9\tabcolsep) * \real{0.10}}\raggedright
% 4\strut
% \end{minipage} &
% \begin{minipage}[t]{(\columnwidth - 9\tabcolsep) * \real{0.10}}\raggedright
% Vancouver\strut
% \end{minipage} &
% \begin{minipage}[t]{(\columnwidth - 9\tabcolsep) * \real{0.10}}\raggedright
% BC\strut
% \end{minipage} &
% \begin{minipage}[t]{(\columnwidth - 9\tabcolsep) * \real{0.10}}\raggedright
% RR8\strut
% \end{minipage} &
% \begin{minipage}[t]{(\columnwidth - 9\tabcolsep) * \real{0.10}}\raggedright
% Factory\strut
% \end{minipage} &
% \begin{minipage}[t]{(\columnwidth - 9\tabcolsep) * \real{0.10}}\raggedright
% 1.5\strut
% \end{minipage} &
% \begin{minipage}[t]{(\columnwidth - 9\tabcolsep) * \real{0.10}}\raggedright
% Certified\strut
% \end{minipage} &
% \begin{minipage}[t]{(\columnwidth - 9\tabcolsep) * \real{0.10}}\raggedright
% 2012\strut
% \end{minipage} &
% \begin{minipage}[t]{(\columnwidth - 9\tabcolsep) * \real{0.10}}\raggedright
% 5\strut
% \end{minipage} &
% \begin{minipage}[t]{(\columnwidth - 9\tabcolsep) * \real{0.10}}\raggedright
% 10500\strut
% \end{minipage}\tabularnewline
% \begin{minipage}[t]{(\columnwidth - 9\tabcolsep) * \real{0.10}}\raggedright
% 5\strut
% \end{minipage} &
% \begin{minipage}[t]{(\columnwidth - 9\tabcolsep) * \real{0.10}}\raggedright
% Toronto\strut
% \end{minipage} &
% \begin{minipage}[t]{(\columnwidth - 9\tabcolsep) * \real{0.10}}\raggedright
% ON\strut
% \end{minipage} &
% \begin{minipage}[t]{(\columnwidth - 9\tabcolsep) * \real{0.10}}\raggedright
% GH-50\strut
% \end{minipage} &
% \begin{minipage}[t]{(\columnwidth - 9\tabcolsep) * \real{0.10}}\raggedright
% Factory\strut
% \end{minipage} &
% \begin{minipage}[t]{(\columnwidth - 9\tabcolsep) * \real{0.10}}\raggedright
% 4.5\strut
% \end{minipage} &
% \begin{minipage}[t]{(\columnwidth - 9\tabcolsep) * \real{0.10}}\raggedright
% Not certified\strut
% \end{minipage} &
% \begin{minipage}[t]{(\columnwidth - 9\tabcolsep) * \real{0.10}}\raggedright
% 2015\strut
% \end{minipage} &
% \begin{minipage}[t]{(\columnwidth - 9\tabcolsep) * \real{0.10}}\raggedright
% --\strut
% \end{minipage} &
% \begin{minipage}[t]{(\columnwidth - 9\tabcolsep) * \real{0.10}}\raggedright
% 12000\strut
% \end{minipage}\tabularnewline
% \begin{minipage}[t]{(\columnwidth - 9\tabcolsep) * \real{0.10}}\raggedright
% 6\strut
% \end{minipage} &
% \begin{minipage}[t]{(\columnwidth - 9\tabcolsep) * \real{0.10}}\raggedright
% Montréal\strut
% \end{minipage} &
% \begin{minipage}[t]{(\columnwidth - 9\tabcolsep) * \real{0.10}}\raggedright
% QC\strut
% \end{minipage} &
% \begin{minipage}[t]{(\columnwidth - 9\tabcolsep) * \real{0.10}}\raggedright
% --\strut
% \end{minipage} &
% \begin{minipage}[t]{(\columnwidth - 9\tabcolsep) * \real{0.10}}\raggedright
% School\strut
% \end{minipage} &
% \begin{minipage}[t]{(\columnwidth - 9\tabcolsep) * \real{0.10}}\raggedright
% --\strut
% \end{minipage} &
% \begin{minipage}[t]{(\columnwidth - 9\tabcolsep) * \real{0.10}}\raggedright
% Not certified\strut
% \end{minipage} &
% \begin{minipage}[t]{(\columnwidth - 9\tabcolsep) * \real{0.10}}\raggedright
% 2010\strut
% \end{minipage} &
% \begin{minipage}[t]{(\columnwidth - 9\tabcolsep) * \real{0.10}}\raggedright
% --\strut
% \end{minipage} &
% \begin{minipage}[t]{(\columnwidth - 9\tabcolsep) * \real{0.10}}\raggedright
% 8000\strut
% \end{minipage}\tabularnewline
% \bottomrule
% \end{longtable}

\begin{enumerate}
\def\labelenumi{(\alph{enumi})}
\item
  List the features that you would remove from the dataset before using
  it for model fitting. Give valid \textbf{reasons} for your answer.
\item
  List the features that need to be encoded in this dataset and outline
  which encoding schemes your would use in each case. Give valid
  \textbf{reasons} for your answer.
\item
  What type of an imputation scheme would make sense for the missing
  values in each of the fields \texttt{Year} and \texttt{Mileage}? Give
  valid \textbf{reasons} for your answer.
\item
  You propose to use the simplest sparsity-based method to select the
  best features among those given in the dataset. In just two lines,
  describe the high-level procedure to do this.
\end{enumerate}

    \begin{enumerate}
\def\labelenumi{(\alph{enumi})}
\tightlist
\item
\end{enumerate}

    I would remove Vehicle model as it is information that is likely to
suffer from high dimensionality and will add noise to our predictions
and increase the runtime of our model training. I would also remove City
since it will likely be highly correlated with Province and
multicollinearity will occur.

    \begin{enumerate}
\def\labelenumi{(\alph{enumi})}
\setcounter{enumi}{1}
\tightlist
\item
\end{enumerate}

    Once we remove the features mentioned above, we will one hot encode
Province and Category and we will label encode Certification. We will
one-hot encode Prov and Cat since these categorical variables cannot be
ordered in any meaningful way relative to the target variable whereas
certification can be ordered. Price seems to be positively correlated
with the presence of certification in buses (as observed from the small
subset of data).

    \begin{enumerate}
\def\labelenumi{(\alph{enumi})}
\setcounter{enumi}{2}
\tightlist
\item
\end{enumerate}

    Year: We will impute by the mode year per Category. We can't use mean or
median since this will likely produce decimal values. Tourism buses will
probably want to update their buses frequently and so they will have
more recent buses whereas factory buses are less concerned with the
appearance and functions of their buses and more concerned with saving
money, making a category grouping somewhat appropriate.

Mileage: We will impute by group average mean, where we group the
observations by year. This is reasonable since newer cars/buses are
generally more fuel efficient.

    \begin{enumerate}
\def\labelenumi{(\alph{enumi})}
\setcounter{enumi}{3}
\tightlist
\item
\end{enumerate}

    We will implement Lasso feature selection. Lasso will set some feature
coefficients to 0 to reduce the number of correlated variables, then it
will find the coefficients that minimize the objective (penalized by a
regularization term lambda)

    \begin{tcolorbox}[breakable, size=fbox, boxrule=1pt, pad at break*=1mm,colback=cellbackground, colframe=cellborder]
\prompt{In}{incolor}{ }{\boxspacing}
\begin{Verbatim}[commandchars=\\\{\}]

\end{Verbatim}
\end{tcolorbox}


    % Add a bibliography block to the postdoc
    
    
    
\end{document}
